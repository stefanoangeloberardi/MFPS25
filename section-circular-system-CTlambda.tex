
\section{The Circular Simply Typed $\lambda$-Calculus $\CTlambda$}
We introduce a condition call \emph{global trace condition} (GTC) for any proof 
$\Pi$ that the term $t$ is well-typed. GTC implies a termination result (see below). 
As we explained in the introduction,
GTC was already defined for infinitary simply typed combinatorial terms, 
our contribution is to give an explicit formulation of GTC in the case of the binder $\lambda$.

In this section we also define the set $\GTC$ of well-typed terms 
$t \in \LAMBDA$ satisfying the global trace condition, for some proof 
$\Pi:\Gamma \vdash t:A$, some context $\Gamma$ and some type $A$. 
We will prove that closed terms of $\GTC$ define a set of total functionals:
they include all functionals of G\"{o}del's system $\systemT$,
and we conjecture they define exactly this set of functionals.

We are particulary interested to the set $\CTlambda$ of regular terms of $\GTC$, namely
to a set of terms defined by $\CTlambda = \GTC \cap \Reg$.
The reason for this interest is that $\CTlambda$ is a decidable subset of $\Reg$
and that all terms of $\CTlambda$ (actually, of $\Reg$) can be represented by finite graphs.
For the terms of $\CTlambda$ we will prove that all reduction strategies
%, church-rosser for the \quotationMarks{safe} part of a term, 
on a closed term of type $\N$ strongly normalize to a numeral.
As a consequence, all closed terms $\CTlambda$ are total functionals. 
\\

By building over the $\N$-argument connection (Def. \ref{definition-connection})
we now define the Global Trace Condition and the set of terms $\GTC$  and $\CTlambda$.
Informally, GTC requires that all infinite branches have some trace of some argument
$\N$ passing infinitely many times through the right-hand-side of a $\cond$.

\begin{definition}($\GTC$ and $\CTlambda$)
\label{definition-global-trace-condition}
%\mbox{}
% \linebreak 
Assume $\Pi:\Gamma \vdash t :A$ is a typing proof, $\pi = (\theta_1, \ldots, \theta_n, \ldots)$ 
is a path of $\Pi$ and $\tau =( (k_m,\theta_m), \ldots, (k_n,\theta_n), \ldots)$ 
is a trace of a path $\pi = (\theta_1, \ldots, \theta_n, \ldots)$. 
Assume $i$ is an index of $\pi$.

\begin{enumerate}
\item
$\tau$ is progressing in the index $i$ if and only if: $\theta_i$ is labeled by some $\cond$-rule
and by some judgement $\Gamma \vdash \cond(f,g):\N \rightarrow A$),
$\theta_{i+1}$ is the right child of $\theta_i$
labeled by the judgement $\Gamma \vdash g:\N \rightarrow A$,
and $k_i$ is the index of the first \emph{unnamed} argument of $\theta_i$.
Otherwise $\tau$ is not progressing in $i$.

\item
$t$ satisfies the global trace condition if $t \in \WTyped$
and for some typing proof $\Pi$ of $t$, for all infinite paths $\pi$ of $\Pi$,
there is some infinitely progressing path $\tau$ in $\pi$. $\GTC$ is the set of 
well-typed terms $t \in \LAMBDA$ satisfying the global trace condition.

\item
$\CTlambda = \GTC \cap \Reg$: the cyclic $\lambda$-terms of 
$\CTlambda$ are all well-typed terms which are regular trees 
(having finitely many subtrees),  and which satisfy the global trace condition.

\end{enumerate}
\end{definition}

Global Trace Condition is decidable in polynomial space in the number of
subterms of $t$, counted up to isomorphism, 
because it is a particular case of Neil Jones Size Change algorithm for Program 
Termination (\cite{SCT}). 
We believe that the global trace condition is decidable quite fast
for infinitary $\lambda$-terms describing realistic programs. 

%Another nice feature of the global trace condition is 
%that for $t \in \GTC$ we require that for any proof $\Pi:\Gamma \prove t:A$ 
%all infinite paths have some infinite progressing trace. This proof is almost-
%left-finite, because all leftmost paths from any subterm of $t$ 
%have no progress point and 
%therefore are finite.  We conclude that if $t \in \GTC \subseteq \WTyped$
%then we can assign a unique type to $t$ with a unique proof, which is almost-left-
%finite.

We include now some examples of  terms in $\CTlambda$, i.e., 
of cyclic $\lambda$-terms. We recall that they have to satisfy $\GTC$, 
in particular they are almost-left-finite, and by definition of $\CTlambda$ they are regular.

%18:05 03/06/2024

%\section{Examples of terms of $\CTlambda$}
%21:25 25/03/2025
%15:35 26/03/2025
\subsection{The sum map}

A first example of term of  $\CTlambda$. 
We provide an infinite regular term $\Sum$ computing the sum on $\N$.
In this example, the type superscript $\N$ of variables $x^\N$, $z^\N$ is omitted.

\begin{Eg}
\label{example-sum}
We set $\Sum \sim \lambda x.\cond(x,\lambda z.\Succ(\Sum(x)(z)))$.
$\Sum$ is a regular term because it has finitely many subterms up to isomorphism: 
\begin{center}
  $\Sum$,
  \quad
  $\cond(x,\lambda z.\Succ(\Sum(x)(z)))$,
  \quad
  $x$,
  \quad
  $\lambda z.\Succ(\Sum(x)(z))$,
 \quad
  $\Succ(\Sum(x)(z))$,
  \quad
  $\Sum(x)(z)$,
  \quad
  $\Sum(x)$,
  \quad
   z
\end{center}
$\Sum$ is well-typed by the following derivation.\footnote{Remark that without the 
$\weak$-rule, the context of the top $\golddagger$ would be $x^\N, y^\N$ 
while the context of the bottom $\golddagger$ would be $\emptyset$. 
Without the $\weak$-rule, this particular proof would not be correct.}
\[
\infer[\lambda]{
  \vdash \Sum:\N, 
  %\goldN 
  \goldN
  \rightarrow \N 
   \quad (\golddagger)
}{
  \infer[\cond]{
    x : \N \vdash 
    \cond(x,\lambda z.\Succ(\Sum(x)(z))): \goldN \rightarrow \N
     \quad (\goldspadesuit)
  }{
    \infer[\var]{
      x : \N \vdash x : \N
    }{}
    &
    \infer[\lambda]{
      x:\N \vdash \lambda z.\Succ(\Sum(x)(z)): \goldN \rightarrow \N  
      %\quad (\goldspadesuit)
    }{
      \infer[\Succ]{
        x:\N, z : \goldN 
        \vdash \Succ(\Sum(x)(z)): \N  
        % \quad (\goldspadesuit)
      }{
        \infer[\apvar]{
          x:\N, z : \goldN 
          \vdash \Sum(x)(z): \N
        }{
          \infer[\apvar]{
            x:\N,  z : \N
            \vdash \Sum(x): \goldN \rightarrow \N
          }{
            \infer[\weak]{
              x:\N,  z : \N
              \vdash \Sum: \N, \goldN \rightarrow  \N
            }{
              \infer*{\vdash \Sum: \N, \goldN \rightarrow \N 
                \quad (\golddagger)}{}
            }
          }
        }
      }
    }
  }
}
\]
In a computation we can represent this infinite tree by a finite cyclic graph
by adding a back edge from the  $\golddagger$ in the top to the $\golddagger$ in the 
bottom. 
\end{Eg}

We %colored in \bfColor{oldgold}{old gold} one sequence of atoms $\goldN$:
outlined one trace (one sequence of connected $\N$-arguments)
with $\goldN$. The outlined trace is the unique 
infinite trace, it is cyclic and it is infinitely progressing in the unique infinite path
of $\Sum$.
We marked $\goldspadesuit$ the progress point of the unique infinite trace.
The progress point is repeated infinitely many times in the unique infinite trace,
therefore $\Sum \in \CTlambda$. 


%
%\begin{tikzpicture}
%  %\draw [help lines] (-3,-1) grid (9,7);
%  \coordinate (a) at (0,0) node at (a) {A};
%  \coordinate (c) at (0,5) node at (c) {C};
%  \draw (0,0) -- (0:2cm);
%  \draw (0,0) -- (30:3cm);
%  \draw (0,5) -- +(0:2cm);
%\end{tikzpicture}


%10:43 16/04/2024

\subsection{The Iterator}

\begin{Eg}
We define a term $\Iter$ of  $\CTlambda$ computing the iteration of maps on $\N$.
The term $\Iter$ is a closed normal term of type $(\N \rightarrow \N), \N,\N \rightarrow \N$ such 
that $\Iter(f,a,n)=f^n(a)$ for all numeral $n \in \Num$. 
Let $\Gamma = (f:\N \rightarrow \N, a:\N)$. We write $\approx$ for
the existence of a common reduct in the context $\Gamma$. 
We have to solve the following equations up to $\approx$:

\begin{enumerate}
\item
$\Iter(f,a,0) \approx a$ 
\item
$\Iter(f,a,\Succ (t)) \approx f(\Iter(f,a,t))$
\end{enumerate}

where $f$, $a$ abbreviate $f^{\N\rightarrow\N}$ and $a^\N$.
We solve the equations above with $\Iter \sim \lambda f, a.\iter$
where $\sim$ is the term isomorphism and
$$
\iter \sim \cond (a, \lambda x.f(\iter(x))):\N \rightarrow \N
$$ 
%is a term in the context $\Gamma = (f:\N \rightarrow \N, a:\N)$.
\end{Eg}

\begin{proposition}[The iterator $\Iter$ is a term of $\CTlambda$]
\label{proposition-iterator-in-CT-lambda}
$\iter \in \Reg \cap \WTyped \cap \GTC = \CTlambda$, 
hence $\Iter \in \CTlambda$.
\end{proposition}

%\begin{proof}
%The term $\Iter$ is well-typed and regular by definition. 
%We check the global trace condition. 
%\\
%We follow the unique infinite trace $\tau$ of the last unnamed argument $\N$ of $\Iter$ 
%in the unique infinite path $\pi$ of $\Iter$. 
%The trace $\tau$ moves from  $\Iter = \lambda f. \lambda a.\iter$
% to the first unnamed argument of the sub-term $\lambda a.\iter$, 
%then to $a:\N$ in the context of $\iter = \cond (a, \lambda x.f(\iter(x)))$.
%Then the infinite path $\pi$ and moves in this order to:
% $\lambda x.f(\iter(x))$, $f(\iter(x))$, $ \iter(x)$, $\iter$
%In the meanwhile, $\tau$ progresses from $\cond$ to the second argument $\lambda x.f(\iter(x))$
%of $\cond$, then moves to $x$ in the context of $f(\iter(x))$,
%then to $x$ in the context $\iter(x)$, and eventually to $x$ in the context of $\iter$.
%%10:30 24/03/24
%In this moment the infinite path $\pi$ loops from $\iter$ to $\iter$. At each loop the trace $\tau$ 
%progresses once. We conclude that $\tau$ infinitely progresses.
%\end{proof}

The proof of $\Gamma \vdash \iter: \N \rightarrow \N$ 
includes a type inference from the term $\iter$ to itself.
We draw the type inference in the picture below and we outline
with $\goldN$ the unique infinitely progressing trace
(the infinite traces of $f:\N \rightarrow \N$ and $a:\N$ do no progress). 
If we want a finite cyclic graph notation we add a back edge from the 
$\golddagger$ on the top to the $\golddagger$ in the bottom.
We mark $\goldspadesuit$ the progress point, which is cyclically repeated in 
the outlined infinite trace. 
\[
%\infer[\lambda]{ %opening: \infer[\lambda]
%  \vdash \Iter:(\N \rightarrow \N), \N, \goldN \rightarrow \N
% }{
%  \infer[\lambda]{ %opening: \infer[\lambda]
%  f:\N \rightarrow \N
%  \vdash \lambda a.\iter:\N, \goldN \rightarrow \N
%  }
{
    \infer[\cond]{ %opening: \infer[\cond]
      \Gamma 
      \vdash \iter: \goldN \rightarrow \N 
        \quad (\goldspadesuit, \golddagger)
     }{ 
         \infer[\var]{
       \Gamma 
      \vdash a:\N}{}
     &
        {\ \ \ \ \ \ }
        {\infer[\lambda] %opening: \infer[\lambda]
         {
         \Gamma
          \vdash \lambda x.f(\iter(x)):\goldN \rightarrow \N
         }{
         \infer[\ap]{ %opening: \infer[\ap]
           \Gamma, x:\goldN
          \vdash f(\iter(x)):\N
           }{
          \infer[\var]{
       \Gamma, x:\N 
      \vdash f:\N \rightarrow \N}{}
           {\ \ \ \ \ \ \ \ \ \ \ \ }
           {\infer[\apvar] %opening: \infer[\apvar]
            {\Gamma, x:\goldN
        \vdash \iter(x): \N 
             }{
          \infer[\weak]{\Gamma, x:\N
                                 \vdash \iter: \goldN \rightarrow \N}
                                {\infer*{\Gamma
                                 \vdash \iter: \goldN \rightarrow \N
                                  \ \ \ ( \golddagger)}{}
             }
           }
          }
        }%closing: \infer[\apvar]
      }%closing: \infer[\ap]
    }%closing: \infer[\lambda]
   }%closing: \infer[\cond]
 }%closing: \infer[\lambda]
%}%closing: \infer[\lambda]
\]




\subsection{The Interval Map}
A third example. We simulate lists with one type variable $\alpha$ 
and two variables $\nil:\alpha$ and $\cons:\N,\alpha \rightarrow \alpha$. 
We recursively define a notation for lists by $[]=\nil$,
$a @ l=\cons(a,l)$ and $[a,\vec{a}] = a @ [\vec{a}]$. We add no elimination rules 
for lists, though, only the variables $\nil$ and $\cons$. Elimination rules are not 
required in our example.

\begin{Eg}
We will define a term $\Interval$ with one argument $f:\N \rightarrow \N$ and 
three argument $a,x,y:\N$ (we skip all type superscripts), such that 
\[
\Interval(f,a,n,m) \  \approx \ [f^n(a), f^{n+1}(a), \ldots, f^{n+m}(a)] \  : \ \alpha
\]
for all numeral $n,m \in \Num$. We have to solve the recursive equations, where
$\approx$ denotes having a common reduct:
\begin{center}
  $\Interval(f,a,n,0) \approx [f^n(a)]$
  \quad
  and
  \quad
  $\Interval(f,a,n,\Succ (m))  \approx f^n(a) @ \Interval(f,a,\Succ(n),m)$.
\end{center}
Assume $\iter = \cond (a, \lambda x.f(\iter(x))):\N \rightarrow \N$ is the term
in the context $\Gamma = \Sigma,(f:\N \rightarrow \N, a:\N)$, where
$\Sigma = (\nil:\alpha, \cons:\N,\alpha \rightarrow \N)$,
and
$(f:\N\rightarrow\N, a:\N)$ is the context for $\iter$ defined 
in the previous sub-section. In particular, $\iter(n) \approx f^n(a) : \N$ for all $n \in \Num$.
We solve the recursive equation for $\Interval$ with $\Interval = \lambda f,a.\interval$, where 
\[
\interval:\N,\N \rightarrow \alpha
\]
is a term in the context $(f:\N\rightarrow\N, a:\N)$ defined by 
\[
\interval 
\ \ \ \sim \ \ \ 
\lambda x.\cond (\base,  \lambda y.j),
\]
where the base case and the inductive case are
\[
\base 
\ \  \sim \ \   
[\iter(x)]
%\ \ \ =\ \ \ 
%\cons(\iter(x),\nil)
\ \ \  \ \ \ 
j 
\ \   \sim \ \ 
\iter(x) @ \interval(\Succ(x))(y)
%\ \ \ = \ \ \  
%\cons(\ \iter(x), \ \interval(\Succ(x))(y) \ )
\]
\end{Eg}

\begin{proposition}[The interval map $\Interval$ is a term of $\CTlambda$]
 \label{proposition-interval-in-CT-lambda}
$\Interval \in \Reg \cap \WTyped \cap \GTC = \CTlambda$.
\end{proposition}

%\begin{proof}
%The term $\Interval$ is well-typed and regular by definition. We check the global trace condition.
%Any infinite path $\pi$ either moves to $\iter$, for which we already checked the global trace condition,
%or cyclically moves from $\interval$ to $\interval$.
%We follow the unique infinite 
%trace $\tau$ of the last unnamed argument $\N$ of $\Interval$ inside this infinite path.
%The trace $\tau$ moves to the last unnamed argument $\N$ of  
%$\interval:\N,\N \rightarrow \N$, then to the last unnamed argument of
%$\cond (\ [\iter(x)],  \  \lambda y.\iter(x) @ (\lambda x.\interval)(\Succ(x))(y) \ )$.
%In this step $\tau$ progresses, and moves to 
%the first unnamed argument of $\lambda y.\iter(x) @ (\lambda x.\interval)(\Succ(x))(y) \ )$,
%then to $y:\N$ in the context of $\iter(x) @ (\lambda x.\interval)(\Succ(x))(y) \ )$.
%After one $\apvar$ rule, the trace $\tau$ reaches the unique unnamed argument of 
%$(\lambda x.\interval)(\Succ(x))$, then the last unnamed argument $\tau$ of $\interval$. 
%From $\interval$ the trace $\tau$ loops. Each time $\tau$ moves from $\interval$ to $\interval$
%then $\tau$ progresses once. We conclude that $\tau$ infinitely progresses.
%\end{proof}
%14:27 24/04/2024

The proof above includes a type inference from the term $\interval$ to itself.
We draw the type inference in the figure \ref{figure-term-interval}
and we outline with $\goldN$ the unique infinitely progressing trace. 
If we want a graph notation, we add a back edge from the 
top $\golddagger$ to the bottom $\golddagger$. We need the $\weak$-rule
to remove the variables $x$, $y$ which are no more used in the top $\golddagger$
and do not occur in the bottom $\golddagger$. 
We mark $\goldspadesuit$ the progress point which is cyclically repeated in 
the outlined infinite trace. 
\\

% TYPING PROOF 
\begin{figure}
\label{figure-term-interval}

\begin{center}
  
  \begin{footnotesize}
%\[
%\infer[\lambda]
% {\Sigma\vdash \Interval:(\N \rightarrow \N),\N,\N,\goldN\rightarrow\alpha}
% {\infer[\lambda]
%   {\Sigma,f:\N\rightarrow\N \vdash \lambda a.\interval:\N,\N,\goldN   
%      \rightarrow\alpha}
   {\infer[\lambda]  
     {\Gamma \vdash \interval:\N,\goldN\rightarrow\alpha 
       \ \ \ (\golddagger) }
       {\infer[\cond]{\Gamma, x:\N 
	\vdash 
	\cond (\base,  \inductive )
	:\goldN\rightarrow\alpha \ \ \ (\goldspadesuit) } 
       {\infer[\apvar]{\Gamma, x:\N 
	           \vdash 
	           \base:\alpha}
             {\infer[]{\Gamma, x:\N 
	           \vdash 
	           \cons(\iter(x)):\alpha \rightarrow\N}{\ldots}}
            {\ \ \ \ \ \ }  &
          \infer[\lambda]{\Gamma, x:\N 
	           \vdash 
	           \lambda y.j : \goldN\rightarrow\alpha}
             {\infer[\apnotvar]{\Gamma, x:\N, y:\goldN \vdash
               j : \alpha}
            {\infer[]
                     {\Gamma, x:\N, y:\N 
                          \vdash \iter(x):\N}{\ldots}
               {\ \ \ \ \ \ }
                     {\infer[\apvar]{\Gamma, x:\N, y:\goldN 
                          \vdash \interval(\Succ(x))(y):\alpha}
                         {\infer[\apnotvar]{\Gamma, x:\N, y:\N 
                          \vdash \interval(\Succ(x)):\goldN \rightarrow\alpha}
                              {\infer[\weak]{\Gamma, x:\N, y:\N 
                          \vdash \interval:\N,\goldN \rightarrow\alpha}
                                {\infer[]{\Gamma 
                          \vdash \interval:\N,\goldN \rightarrow\alpha
                           \ \ \ (\golddagger)}
              {\ldots} } {\ldots\ldots}  }}}
             }
           }
         }
       }  
    }
%   }
%\]
  \end{footnotesize}

\mbox{Figure \ref{figure-term-interval}}
\end{center}

\end{figure}
%15:50 26/03/2025

The term $\interval$ is an example of how difficult is to completely normalize
an infinite term. We have infinitely many nested $\beta$-reductions in
$\interval$, one for each $\interval(\Succ (x)) \equiv (\lambda x.\ldots)(\Succ (x))$. 
Suppose we remove all of them in a single step, producing a substitution
$[\Succ (x)/x]$. $\beta$-redexes are nested, therefore 
inside the $\beta$-redex number $k$ we obtain a sub-term $\iter[\Succ (x)/x]
\ldots[\Succ (x)/x]$, with the substitution of $x$ nested $k$ times. 
The result is $\iter[\Succ ^k(x)/x]$.
The nested substitution produce infinitely many pairwise different sub-terms 
$\iter(\Succ ^k(x))$ for all $k \in \N$. 
We need $k$ steps to normalize each $\iter(\Succ ^k(x))$ to $f^k(x)$, 
even if we allow to reduce infinitely many 
$\beta, \cond$-redexes at the same time, because our reduction will consume one 
$\Succ$  at the time. Thus, the total time is infinite, a limit process. 
Besides, the normal form we obtain in this way not regular: it contains terms 
$f^k(x)$ for all $k \in \N$, hence infinitely many pairwise different terms. 

%For this reason we consider normalization as part of the semantics for
%our cyclic graph-notation, except for closed terms of type $\N$, 
%in this case normalization is the computation of the output.

%These infinite sub-terms are of a particulary simple form, though. 
%They are obtained by the repeating $k$ times the assignment $z:=f(z)$, then applying $z:=I$ once
%to the result.

%In conclusion, $\Interval$ is some term of $\CTlambda$ which can be safely 
%normalized, but which cannot be fully normalized in finite time, not even if we 
%allow infinite parallel reductions without any "safety" restriction. The normal form 
%is produced \emph{only in the limit} and it is \emph{not regular}. If we allow to 
%reduce infinitely many nested existing $\beta$-redexes in one step, also the 
%intermediate steps of the infinite reduction of $\Interval$ are not regular.


%16:32 30/04/2024
%22:30 03/06/2024
\subsection{The Ackermann Function (Better Code it is Easier to Analyse)}
The Global Trace Condition is a \quotationMarks{fragile} property: two
terms can be trivial variant each other, can compute the function, yet
one term can satisfy the global trace condition while the other term does not.
As an example we consider Péter and Robinson version of Ackermann function. We can write it in two ways, as:
  \[
  \AckA = \lambda \redm, \bluen.\Cond{\Suc{\bluen}}{\lambda m'.\Cond{\AckA(m',1)}{\lambda n'.\AckA(m',\AckA(\mathbf{\Suc{m'}},n'))}(\bluen)}(\redm)
  \]
and as 
  \[
  \AckB = \lambda \redm, \bluen.
\Cond{\Suc{\bluen}}{\lambda m'.\Cond{\AckB(m',1)}{\lambda n'.\AckB(m',\AckB(\redm,n'))}(\bluen)}(\redm)
  \]
The only difference between the two terms is that $\Suc{m'}$ in the first
term becomes $\redm$ in the second term. $\AckA$ and $\AckB$ are 
extensionally equal, because in all computations $m'$ is the predecessor of $m$. 
However, by direct computation we find out that 
$\AckA$ does \emph{not} satisfy the Global Trace condition, while $\AckB$ does
satisfy it. 

The reason is that the value of the term $u= \Suc{m'}$ is already stored in a 
variable $\redm$, so we should write $\ldots (\mathbf{\Suc{m'}},n') \ldots$ in the 
form  $\ldots (\redm,n') \ldots$, for two reasons.
First, it is a waste of time to recompute the value $u= \Suc{m'}$ if this value is 
already stored in $\redm$. Second, if we write $\ldots (\redm,n') \ldots$ instead of 
$\ldots (\Suc{m'},n') \ldots$, then we use the variable $\redm$ as argument of our
term. When we use a variable as argument, way we add more connections and 
more traces to our term, and it is more likely that the Global Trace algorithm finds 
some infinitely increasing trace and therefore proves that the map we defined is 
total. Replacing  a term $u$ with a variable $\redm$, when this replacing is sound, 
provides better code which is easier to prove total automatically.
