% !TEX root = main.tex
\section{Comparison with Previous Works}
\label{section-comparison}

In this section we quote more results obtained about the circular versions of system $\systemT$,
in its combinatorial version. As a future work, we consider the possibility of extending these results
to our circular version of system $\systemT$.

Das compared the descriptive power of circular terms of $\systemT$ 
having degree $n$ type with the descriptive power of terms of original system $\systemT$,
but having degree $n+1$ types. 
He  proved (\cite{2021-Anupam-Das}, \cite{DBLP:conf/fscd/000221})
that the two set of terms define the same type $1$ functions. In the same papers he also proved 
that the circular terms of $\systemT$ and the terms of original system $\systemT$ 
define the same functionals at all types.

Curzi and Das  also developed a circular version of Bellantoni-Cook system for 
polynomial-time functions (\cite{DBLP:conf/lics/Curzi022}), 
then some criterions for identifying polynomial-time functions
in the circular system $\systemT$ (\cite{DBLP:conf/csl/Curzi023}),
and they extended the circular syntax from $\systemT$ to all functions provably total in 
$\Pi^1_2$-second order arithmetic (\cite{DBLP:conf/lics/Curzi023}).
Kuperberg, Pinault and Pous (\cite{2021-Kuperberg-Pinault-Pous})
proved that the contraction-free part of circular system $\systemT$
captures exactly the primitive recursive functions.
