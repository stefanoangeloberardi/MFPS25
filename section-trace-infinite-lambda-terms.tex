\section{The trace of the infinite $\lambda$-terms}\label{section-trace-infinite-lambda-terms}

%19:34 27/03/2024
We define a notion of trace for possibly infinite $\lambda$-terms, 
describing how an input of type $\N$ is used when computing the output
and when its value decreases. The notion of trace is useful to analyse termination.
If, in all infinite computations of a term, we are able to find some infinitely
decreasing value, then no infinite computation can exist for this term. 
The first step toward the notion of trace is defining a notion of \emph{connection} 
between arguments of type $\N$ in a proof that evidences the well-typedness of $t$. 
To this aim, we need to define \emph{lists of argument types} and \emph{indexes of atomic types} for a term.

%We sketch the notion of connection through an example.
%Assume 
%\[
%  {x_1}^{A_1}:A_1,{x_2}^{A_2}:A_2 \vdash t[x_1,x_2]:B_3 \rightarrow \N
%\]
%Then the list of argument types of $t$
%is $A_1, A_2, B_3$. $A_1,A_2$ are arguments with names $x_1, x_2$, while $B_3$ is an unnamed
%argument (it will be denoted by some bound variable in $t$). 
%The index of an $\N$-argument of $t$ is any $j \in \{1,2,3\}$ such that $A_j=\N$
%or $B_j=\N$ respectively: in this case all integers $1,2,3$ are indexes of some $\N$-argument,
%in general we can have argument with type different from $\N$.
%
%Remark that for an open term $ t[x_1,x_2]$ we list among the ``argument types'' also the
%types $A_1, A_2$ of the free variables. We motivate this terminology:
%in a sense, $t$ is an abbreviation of the closed term $t' = \lambda  
%{x_1}^{A_1},{x_2}^{A_2}.t: (  A_1,A_2,B_3 \rightarrow \N )$, and the argument types of $t'$ are
%$A_1, A_2, B_3$ and they include $A_1, A_2$. 
%
%Below is the formal definition of argument types for a term.

\begin{definition}(List of argument types)
Let $J$ be a judgment.
Then $J$ has the form $\Gamma \vdash t:B$,
where $\Gamma = (x_1^{A_1}:A_1,\ldots, x_n^{A_n}:A_n)$ and 
$B = \vec{B}\rightarrow \N$ or $\vec{B}\rightarrow \alpha$ and 
$\vec{B}=B_{1}, \ldots, B_{m}$. 
We define the following notions. 

\begin{enumerate}
\item
The \emph{list of argument types} of the judgment $J$ is $\vec{C} = \vec{A},\vec{B}$,
where $\vec{A} = A_1,\ldots,A_n$.

\item
An element $A_i \in \vec{A}$ is called a \emph{named argument} of $J$, with index $i$.
    
\item
An element $B_j$ of $\vec{B}$ is called an \emph{unnamed argument} of $J$, with
index $j$. 

\item
An \emph{index of an $\N$-argument} 
of $t$ is any $A_i = \N$ and any $B_j = \N$.

\end{enumerate}
\end{definition}

%We now define the connection between $\N$-arguments of subterms of $t$
%in a proof $\Pi:: \Gamma \vdash t:A$ of  $\LAMBDA$. The connection describes
%how an input  moves through the infinite unfolding of the term.
%The definition of  atom connection for a syntax including the binder $\lambda$ 
%is the main contribution of this paper. 
%%Two $\N$-argument are in
%%connection if and only if they receive the same global input: local input are ignored.
%%In many cases two corresponding argument types have the same index, but if we insert or remove
%%free variables or arguments the index may change.
%Before providing the general definition, we discuss the notion of connection through examples. 
%We draw in the same color two $\N$-argument which are in connection. 
%
%\begin{Eg}\label{eg:0}\rm
%An example of  atom connection for some instance of the $\weak$-rule.
%\[
%\infer[(\weak)]{
%  {x_1} : \bfColor{red}{\N},{x_2} : \bfColor{blue}{\N}, x_3:\bfColor{oldgold}{\N}
%  \prove t : \bfColor{orange}{\N} \rightarrow \N
%}{
%  {x_1} : \bfColor{red}{\N},{x_2} : \bfColor{blue}{\N} 
%  \prove t : \bfColor{orange}{\N} \rightarrow \N
%}
%\]
%\end{Eg}
%Remark that the type $\N$ of the variable $x_3$, colored in \bfColor{oldgold}{old gold} and 
%introduced by weakening, is in connection with no type in the rule $\weak$.
%
%%20:13 15/04/2024
%\begin{Eg}\label{eg:1}%\rm
%An example of  atom connection for some instance of the $\apnotvar$-rule.
%We assume that $a$ is \emph{not} a variable.
%\[
%\infer[(\apnotvar)]{
%  x_1 : \bfColor{red}{\N},{x_2} : \bfColor{blue}{\N}
%  \prove f(a) : \bfColor{orange}{\N} \rightarrow \N
%}{
%  x_1 : \bfColor{red}{\N},{x_2} : \bfColor{blue}{\N}
%  \prove f : \bfColor{oldgold}{\N}, \bfColor{orange}{\N} \rightarrow \N
%  &
%  x_1 : \bfColor{red}{\N}, x_2: \bfColor{blue}{\N} \prove a : \N
%}
%\]
%\end{Eg}
%Remark that the first unnamed argument of $f$ (colored in \bfColor{oldgold}{old gold}) 
%is in connection with no argument in the rule $\apnotvar$.
%The reason is that in the term $f(a)$,
%the first argument of $f$ receives a value from the value $a$ which is local to the term $f(a)$,
%does not receive a value from outside the term.
%However, the first argument of $f$ can be in connection with some argument higher in the typing proof. 
%%13:21 15/04/2024
%
%We postpone examples with the rule $\apvar$.
%
%\begin{Eg}\label{eg:2}%\rm
%An example of  atom connection for the rule $\cond$.
%\[
%\infer[(\cond)]{
%  {x_1} : \bfColor{red}{\N},{x_2} : \bfColor{blue}{\N}
%  \prove \cond(f,g) : \bfColor{oldgold}{\N} \rightarrow \N
%}{
%  x_1 : \bfColor{red}{\N},{x_2} : \bfColor{blue}{\N} \prove f : \N
%  &
%  x_1:\bfColor{red}{\N}, x_2:\bfColor{blue}{\N} \prove g:\bfColor{oldgold}{\N} \rightarrow \N
%}
%\]
%\end{Eg}
%Remark that the first unnamed argument of $\cond(f,g)$ (colored in \bfColor{oldgold}{old gold}) 
%is in connection the type of the first unnamed one of $g$ in the second premise,
%but it has no connection with the first premise. 
%
%
%\subsection{A formal definition of connection}

%15:19 24/03/2025
%
%Let $r$ be an instance of a typing rule whose conclusion is $\Gamma \vdash t:\vec{B} \rightarrow \N$
%where the length of $\Gamma$ is $n$, 
%and premises $J_1,\ldots,J_k$ ($k=0,1,2$). 
%We define a \emph{connection} over $r$ that is a binary relation
%between the arguments of the conclusion $\Gamma \vdash t:\vec{B} 
%\rightarrow \N$ of $r$ and the argument of any premise $\Gamma' \vdash t':\vec{B'} 
%\rightarrow \N$ of $r$. 

Assume $\theta$ is a node of a proof decorated with a sequent $\Gamma \vdash t:A$
and a rule $r$. For each argument $k$ in the sequent $\Gamma \vdash t:A$ decorating
$\theta$, for each argument  $k'$ in a sequent $\Gamma' \vdash t':A'$decorating
any premise $\theta'$ of $\theta$ we define
the relation: ``$k',\theta'$ connected to $k,\theta$". The definition is by cases on $r$.
We first list connections for the rules: application, $\lambda$-abstraction, weakening and $\var$.

\begin{definition}(Connection in a proof of  $\LAMBDA$: application)
\label{definition-connection-application}
Suppose $\Gamma = x_1^{A_1}:A_1, \ldots, x_n^{A_n}:A_n$ and $B = B_1 \rightarrow \ldots \rightarrow B_m \rightarrow a$, with $a=\N$ or $a$ variable type.
\begin{enumerate}

\item
Suppose $\Gamma \vdash f(x_i^{A_i}) : B$ is obtained by a rule $\apvar$ from
$\Gamma \vdash f: A_i \rightarrow B$. 
Then:
\begin{enumerate}
\item
for $j=1, \ldots, n$ 
the $j$-th named argument $x_j^{A_j}$ in the conclusion is connected to the $j$-th named 
argument $x_j^{A_j}$ in the assumption, 
\item
the $i$-th named argument $x_i^{A_i}$ in the conclusion is also connected to the first unnamed
argument $A_i$ in the assumption,
\item
for $k=1, \ldots, m$ the $k$-th unnamed argument $B_k$ in the conclusion 
is connected to the $(k+1)$-unnamed argument $B_k$ in the assumption.
\end{enumerate}
\begin{prooftree}
\def\extraVskip{2pt}
\def\ScoreOverhang{0pt}
\AxiomC{$ \ldots \vnode1{x_i^{A_i}} \ldots \vnode2{x_j^{A_j}}  \ldots 
\vdash f: \vnode3{A_i }
\rightarrow \ldots \rightarrow \vnode4{B_k} \rightarrow \ldots \rightarrow a $}
\AxiomC{}
\BinaryInfC{$ \ldots, \vnode5{x_i^{A_i}} \ldots \vnode6{x_j^{A_j}} \ldots  \vdash f(x_i^{A_i}) 
: \ldots \rightarrow \vnode7{B_k} \rightarrow \ldots \rightarrow a  $}
\end{prooftree}

% DRAW EDGES
%bent edges. Syntax: source/target/bending degree (can be negative)
$
\bentdirflowedges{node5/node1/60}   
\bentdirflowedges{node5/node3/-45}  
\bentdirflowedges{node6/node2/60}
\bentdirflowedges{node7/node4/60}
$    
\item
Suppose $\Gamma \vdash f(a) : B$ is obtained by a rule $\ap$ from
$\Gamma \vdash f: A \rightarrow B$ and $\Gamma \vdash a: A$. 
Then:
\begin{enumerate}
\item
for $j=1, \ldots, n$ 
the $j$-th named argument $x_j^{A_j}$ in the conclusion is connected to the $j$-th named 
argument $x_j^{A_j}$ in both assumptions.
\item
for $k=1, \ldots, m$ the $k$-th unnamed argument $B_k$ in the conclusion 
is connected to the $(k+1)$-unnamed argument $B_k$ in the first assumption.
\end{enumerate}
\begin{prooftree}
\def\extraVskip{2pt}
\def\ScoreOverhang{0pt}
\AxiomC{$ \ldots \vnode1{x_j^{A_j}} \ldots \vdash f: A
\rightarrow \ldots \rightarrow \vnode2{B_k} \rightarrow \ldots \rightarrow a $}
\AxiomC{$ \ \ \ \ \ \  \ \ \  \ldots \vnode3{x_j^{A_j}} \ldots \vdash a: A$}
\BinaryInfC{$ \ldots \vnode5{x_j^{A_j}} \ldots  \vdash f(a) 
: \ldots \rightarrow \vnode6{B_k} \rightarrow \ldots \rightarrow a  $}
\end{prooftree}

% DRAW EDGES
%bent edges. Syntax: source/target/bending degree (can be negative)

$
\bentdirflowedges{node5/node1/30}   
\bentdirflowedges{node5/node3/30}  
\bentdirflowedges{node6/node2/45}
$    

\item
Suppose $\Gamma \vdash \lambda x^A.v$ is obtained by a rule $\lambda$
from $\Gamma \setminus \{ {x}^{A} : A \} \vdash v : \N$.
Then:

\begin{enumerate}
\item
the first unnamed argument $A$ in the conclusion 
is connected to the last named argument $x^A$ in the assumption,
and for $k=1, \ldots, m$ the $(k+1)$-unnamed argument $B_k$ in the conclusion is
connected to the $k$-unnamed argument $B_k$ in the assumption.

For named arguments we have two sub-cases, according if
${x}^{A} \in \FV(\Gamma)$ or not.

\item
Suppose $x^A = {x_i}^{A_i}$ for some (unique) $i=1, \ldots, n$.
In this case
for $a=1, \ldots, i-1$, $b=i+1, \ldots, n$ 
the $a$-th, $b$-th named argument $x_a^{A_a}$ in the conclusion are connected to 
the $a$-th named 
argument $x_a^{A_b}$ and to the $(b-1)$-th 
named argument $x_b^{A_b}$ in the assumption, respectively.

\begin{prooftree}
\def\extraVskip{2pt}
\def\ScoreOverhang{0pt}
\AxiomC{$ \ldots \vnode1{x_a^{A_a}} 
\ldots \vnode3{x_{b-1}^{A_{b-1}}}  
\ldots  \vnode4{x_i^{A_i} }
\vdash
v: \ldots \rightarrow \vnode5{B_k} \rightarrow \ldots \rightarrow a $}
\AxiomC{}
\BinaryInfC{$ \ldots \vnode6{x_a^{A_a}} 
\ldots x_i^{A_i} \ldots \vnode7{x_{b}^{A_{b}}}
\vdash \lambda x_i^{A_i}.v : \vnode8{A_i }
\rightarrow \ldots \rightarrow \vnode9{B_k} \rightarrow \ldots \rightarrow a $}
\end{prooftree}



% DRAW EDGES
%bent edges. Syntax: source/target/bending degree (can be negative)
$
\bentdirflowedges{node6/node1/60}   
\bentdirflowedges{node7/node3/60}  
\bentdirflowedges{node8/node4/-45}
\bentdirflowedges{node9/node5/60}
$    
\item
Suppose $x^A \not = {x_i}^{A_i}$ for all $i=1, \ldots, n$. 
 In this case
for $j=1, \ldots, n$ 
the $j$-th named argument $x_j^{A_j}$ in the conclusion is connected to the $j$-th named 
argument $x_j^{A_j}$ in the assumption.


\begin{prooftree}
\def\extraVskip{2pt}
\def\ScoreOverhang{0pt}
\AxiomC{$ \ldots \vnode1{x_j^{A_j}} 
\ldots  \vnode4{x^{A} }
\vdash
v: \ldots \rightarrow \vnode5{B_k} \rightarrow \ldots \rightarrow a $}
\AxiomC{}
\BinaryInfC{$ \ldots \vnode7{x_{j}^{A_{j}}}
\vdash \lambda x^{A}.v : \vnode8{A }
\rightarrow \ldots \rightarrow \vnode9{B_k} \rightarrow \ldots \rightarrow a $}
\end{prooftree}

% DRAW EDGES
%bent edges. Syntax: source/target/bending degree (can be negative)
$
\bentdirflowedges{node7/node1/60}  
\bentdirflowedges{node8/node4/0}
\bentdirflowedges{node9/node5/60}
$    


\end{enumerate}

\item
Assume $\Gamma \vdash t : B$ is obtained by a rule $\weak$ from $\Gamma' \vdash t : B$
where $\Gamma' = {x'_1}^{A'_1}:A_1, \ldots, {x'_{n'}}^{A'_{n'}}:A'_{n'}$. Assume
$\phi:\Gamma' \rightarrow \Gamma$ is the map associated to the rule\footnote{
We recall that $\phi:[1,n'] \rightarrow [1,n]$, 
and for all $a = 1, \ldots, n'$, the value $\phi(a) \in [1,n]$ is the unique value 
such that ${x'_{a}}^{A'_{a}} = x_{\phi(a)}^{A_{\phi(a)}}$
(Def. \ref{definition-context-lambda}.\ref{definition-context-lambda-08}).}.
Then:
\begin{enumerate}
\item
for $j=1, \ldots, n$ and $a = 1, \ldots, n'$,
the $j$-th named argument $x_j^{A_j}$ in the conclusion is connected to the $a$-th named 
argument ${x'}_{a}^{A'_{a}}$ in the assumption if and only if $j = \phi(a)$
(i.e., if and only if  $x_j^{A_j} = {x'}_{a}^{A'_{a}}$).

\item
for $k=1, \ldots, m$ the $k$-th unnamed argument $B_k$ in the conclusion 
is connected to the $k$-th unnamed argument $B_k$ in the assumption.
\end{enumerate}

\begin{prooftree}
\def\extraVskip{2pt}
\def\ScoreOverhang{0pt}
\AxiomC{$ \ldots \vnode1{{x'}_{a}^{{A'}_{a}}} \ldots 
\vdash t: \ldots \rightarrow \vnode2{B_k} \rightarrow \ldots \rightarrow a $}
\AxiomC{}
\BinaryInfC{$ \ldots, \vnode3{x_{\phi(a)}^{A_{\phi(a)}}} \ldots \vdash t 
: \ldots \rightarrow \vnode4{B_k} \rightarrow \ldots \rightarrow a  $}
\end{prooftree}

% DRAW EDGES
%bent edges. Syntax: source/target/bending degree (can be negative)
$
\bentdirflowedges{node3/node1/60}   
\bentdirflowedges{node4/node2/60}  
$    

\end{enumerate}

No connection is defined for the rule $\var$.

\end{definition}

Now we list connections for $\cond, \Succ, 0$.

\begin{definition}(Connection in a proof of  $\LAMBDA$: $\cond, \Succ, 0$)
\label{definition-connection-cond}

\begin{enumerate}
\item
Suppose $\Gamma \vdash \cond(f,g): \N \rightarrow B$ is obtained by a rule $\cond$ from
$\Gamma \vdash f: B$ and $\Gamma \vdash g: \N \rightarrow B$.
Then:
\begin{enumerate}
\item
for $j=1, \ldots, n$ 
the $j$-th named argument $x_j^{A_j}$ in the conclusion is connected to the $j$-th named 
argument $x_j^{A_j}$ in both assumptions.
\item
the first unnamed argument $\N$ in the conclusion is connected to the 
the first unnamed argument $\N$ in the second assumption,
and for $k=1, \ldots, m$, the $(k+1)$-th unnamed argument $B_k$ in the conclusion 
is connected to the $k$-unnamed argument $B_k$ in the first assumption,
and to the $(k+1)$-unnamed argument $B_k$ in the first assumption.
\end{enumerate}

\begin{prooftree}
\def\extraVskip{2pt}
\def\ScoreOverhang{0pt}
\AxiomC{$ \ldots \vnode1{x_j^{A_j}} 
\ldots 
\vdash
f :  \ldots \rightarrow \vnode2{B_k} \rightarrow \ldots \rightarrow a $}
\AxiomC{$  \ \ \ \ \ \  \ \ \  
\ldots \vnode3{x_j^{A_j}} \ldots 
\vdash
g :  \vnode4{\tt N} \rightarrow
\ldots \rightarrow \vnode5{B_k} \rightarrow \ldots \rightarrow a$}
\BinaryInfC{$ \ldots \vnode6{x_j^{A_j}} 
\ldots 
\vdash
{\tt cond}(f,g) :   \vnode7{\tt N} \rightarrow \ldots \rightarrow \vnode8{B_k} \rightarrow \ldots \rightarrow a $}
\end{prooftree}

% DRAW EDGES
%bent edges. Syntax: source/target/bending degree (can be negative)
$
\bentdirflowedges{node6/node1/30} 
\bentdirflowedges{node6/node3/30} 
\bentdirflowedges{node7/node4/0}
\bentdirflowedges{node8/node2/30}
\bentdirflowedges{node8/node5/0}
$    


\item
Suppose $\Gamma \vdash \Succ(u):\N$ is obtained by a rule $\Succ$ from
$\Gamma \vdash \Succ(u):\N$.
Then for $j=1, \ldots, n$ 
the $j$-th named argument $x_j^{A_j}$ in the conclusion is connected to the $j$-th named 
argument $x_j^{A_j}$ in the assumption.


\begin{prooftree}
\def\extraVskip{2pt}
\def\ScoreOverhang{0pt}
\AxiomC{$ \ldots \vnode1{x_j^{A_j}} 
\ldots 
\vdash
t: {\tt N} $}
\AxiomC{}
\BinaryInfC{$ \ldots \vnode2{x_{j}^{A_{j}}}
\vdash {\tt S}(t) : {\tt N}$}
\end{prooftree}

% DRAW EDGES
%bent edges. Syntax: source/target/bending degree (can be negative)
$
\bentdirflowedges{node2/node1/60} 
$    


\end{enumerate}

No connection is defined for the rule $0$.

\end{definition}

%Below we include some examples. We draw two connected $\N$-arguments
%with the same color. If we want to highlight a connection we also outline
%the $\N$-arguments. Occurrences of $\N$ in the same color or outlined denote
%the same type, colors denote traces.
%
%
%\begin{Eg}\label{eg:3}%\rm
%Atom connection for $\apvar$, application to a variable. The outlined $\goldNout$ in the conclusion
%is connected to two $\goldNout$s in the premise.
%\[
%\infer[(\apvar)]{
%  x_1 : \redN,{x_2} : \blueN, x  : \goldNout
%  \prove f(x) : \orangeN \rightarrow \N
%}{
%  x_1 : \redN, {x_2} : \blueN, x  : \goldNout
%  \prove f : \goldNout, \orangeN \rightarrow \N
%}
%\]
%\end{Eg}
%
%
%%Remark that the last variable in the conclusion of the rule
%%is in connection with the first unnamed argument of $f$ (colored in \bfColor{oldgold}{old gold}) 
%%in the premise, and with the variable with the same name in the premise: there are
%%two connections.
%
%\begin{Eg}\label{eg:4}%\rm
%Atom connection for  $\lambda$-rule.
%%We assume that the bound variable $x$ is  distinct from $x_1$, $x_2$.
%The first unnamed argument $\goldNout$ of the conclusion
%is connected with the $\goldNout$ at the last of the premise's context. 
%\[
%\infer[(\ap)]{
%  x_1: \redN, x_2: \blueN
%  \prove \lambda x.t : \goldNout \rightarrow \N
%}{
%  x_1: \redN, x_2: \blueN, x:\goldNout \prove t : \N
%}
%\]
%\end{Eg}



%Summing up, when
%we move up in a $\apvar$-rule, the type of some free variable corresponds to the 
%type of the first unnamend argument.
%When  we move up in a $\lambda$-rule, the type of the first unnamend argument
%corresponds to the type of the last free variable.

The connection on $\N$-arguments in a proof $\Pi::\Gamma\vdash t:A$ defines a 
graph $\Graph(\Pi)$ whose nodes are all pairs $(k,\theta)$, with $\theta$ node of $\Pi$ and 
$k$ index of some $\N$-argument of  the judgment labeling $\theta$. 
A trace represents the movement of an input information through the 
infinite unfolding of a tree. Formally, we define a trace as a (finite or infinite) path 
in the graph $\Graph(\Pi)$.

\begin{definition}(Trace for well-typed terms in $\LAMBDA$)
Assume $\Pi$ is any typing proof of $t \in \WTyped$.
\begin{enumerate}
\item
A path of $\Pi$ is any finite or infinite branch $\pi =(\theta_0, \ldots, \theta_n, \ldots)$ of $\Pi$.
\item
Assume $\pi =(\theta_0, \ldots, \theta_n, \ldots)$ is a path of $\Pi$, finite or infinite. 
A finite or infinite \emph{trace} $\tau$ of $\pi$ in $\Pi$ is a list 
$\tau =( (k_m,\theta_m), \ldots, (k_n,\theta_n), \ldots)$ such that for all $i=m,\ldots, n,\ldots$:
\begin{enumerate}
\item
$k_i$ is an index of some $\N$-argument of $\theta_i$
\item
if $i$ and $i+1$ are indexes of $\tau$ then $(k_{i+1},\theta_{i+1})$ 
is connected with $(k_i, \theta_i)$ in $\Pi$.
\end{enumerate}

\end{enumerate}
\end{definition}

The starting point of a trace $\tau$ in a path  $\pi =(\theta_0, \ldots, \theta_n, \ldots)$
in $\Pi$ could be $\theta_m$ for some $m \in \N$, 
and therefore different from the starting point $\theta_0$ of the path $\pi$ to which the 
trace belongs.

%20:50 15/04/2024
%10:35 24/04/2024
%12:28 30/04/2024
%14:39 26/03/2025
