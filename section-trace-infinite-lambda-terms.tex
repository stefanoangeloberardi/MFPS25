
\section{The trace of the infinite $\lambda$-terms}
%19:34 27/03/2024
We define a notion of trace for possibly infinite $\lambda$-terms, 
describing how an input of type $\N$ is used when computing the output.
The first step toward a trace is defining a notion of \emph{connection} between arguments
of type $\N$ in the proof that $t$ is well-typed. 
To this aim, we need the notions of \emph{list of argument
 types} and of \emph{index of atomic types} for a term.

%We sketch the notion of connection through an example.
%Assume 
%\[
%  {x_1}^{A_1}:A_1,{x_2}^{A_2}:A_2 \vdash t[x_1,x_2]:B_3 \rightarrow \N
%\]
%Then the list of argument types of $t$
%is $A_1, A_2, B_3$. $A_1,A_2$ are arguments with names $x_1, x_2$, while $B_3$ is an unnamed
%argument (it will be denoted by some bound variable in $t$). 
%The index of an $\N$-argument of $t$ is any $j \in \{1,2,3\}$ such that $A_j=\N$
%or $B_j=\N$ respectively: in this case all integers $1,2,3$ are indexes of some $\N$-argument,
%in general we can have argument with type different from $\N$.
%
%Remark that for an open term $ t[x_1,x_2]$ we list among the ``argument types'' also the
%types $A_1, A_2$ of the free variables. We motivate this terminology:
%in a sense, $t$ is an abbreviation of the closed term $t' = \lambda  
%{x_1}^{A_1},{x_2}^{A_2}.t: (  A_1,A_2,B_3 \rightarrow \N )$, and the argument types of $t'$ are
%$A_1, A_2, B_3$ and they include $A_1, A_2$. 
%
%Below is the formal definition of argument types for a term.


\begin{definition}[List of argument types of a term]
Assume that $\vec{A} = A_1, \ldots, A_n$, $\vec{B}=B_{n+1}, \ldots, B_{n+m}$, 
$\Gamma = \{\vec{x}:\vec{A}\}$,
and $\Gamma \vdash t: \vec{B} \rightarrow \N$.

\begin{enumerate}
\item
The \emph{list of argument types} of $t$ is $\vec{C} = \vec{A},\vec{B}$. 

\item
$A_1, \ldots, A_n$ are the \emph{named arguments}, with names $x_1, \ldots, x_n$.

\item
$B_{n+1}, \ldots, B_{n+m}$ are the \emph{unnamed arguments}.

\item
An \emph{index of an $\N$-argument} 
of $t$ is any $j \in \{1, \ldots, n+m\}$ such that $C_j = \N$.

\end{enumerate}
\end{definition}

%We now define the connection between $\N$-arguments of subterms of $t$
%in a proof $\Pi: t:\Gamma \vdash A$ of  $\LAMBDA$. The connection describes
%how an input  moves through the infinite unfolding of the term.
%The definition of  atom connection for a syntax including the binder $\lambda$ 
%is the main contribution of this paper. 
%%Two $\N$-argument are in
%%connection if and only if they receive the same global input: local input are ignored.
%%In many cases two corresponding argument types have the same index, but if we insert or remove
%%free variables or arguments the index may change.
%Before providing the general definition, we discuss the notion of connection through examples. 
%We draw in the same color two $\N$-argument which are in connection. 
%
%\begin{Eg}\label{eg:0}\rm
%An example of  atom connection for some instance of the $\weak$-rule.
%\[
%\infer[(\weak)]{
%  {x_1} : \bfColor{red}{\N},{x_2} : \bfColor{blue}{\N}, x_3:\bfColor{oldgold}{\N}
%  \prove t : \bfColor{orange}{\N} \rightarrow \N
%}{
%  {x_1} : \bfColor{red}{\N},{x_2} : \bfColor{blue}{\N} 
%  \prove t : \bfColor{orange}{\N} \rightarrow \N
%}
%\]
%\end{Eg}
%Remark that the type $\N$ of the variable $x_3$, colored in \bfColor{oldgold}{old gold} and 
%introduced by weakening, is in connection with no type in the rule $\weak$.
%
%%20:13 15/04/2024
%\begin{Eg}\label{eg:1}%\rm
%An example of  atom connection for some instance of the $\apnotvar$-rule.
%We assume that $a$ is \emph{not} a variable.
%\[
%\infer[(\apnotvar)]{
%  x_1 : \bfColor{red}{\N},{x_2} : \bfColor{blue}{\N}
%  \prove f(a) : \bfColor{orange}{\N} \rightarrow \N
%}{
%  x_1 : \bfColor{red}{\N},{x_2} : \bfColor{blue}{\N}
%  \prove f : \bfColor{oldgold}{\N}, \bfColor{orange}{\N} \rightarrow \N
%  &
%  x_1 : \bfColor{red}{\N}, x_2: \bfColor{blue}{\N} \prove a : \N
%}
%\]
%\end{Eg}
%Remark that the first unnamed argument of $f$ (colored in \bfColor{oldgold}{old gold}) 
%is in connection with no argument in the rule $\apnotvar$.
%The reason is that in the term $f(a)$,
%the first argument of $f$ receives a value from the value $a$ which is local to the term $f(a)$,
%does not receive a value from outside the term.
%However, the first argument of $f$ can be in connection with some argument higher in the typing proof. 
%%13:21 15/04/2024
%
%We postpone examples with the rule $\apvar$.
%
%\begin{Eg}\label{eg:2}%\rm
%An example of  atom connection for the rule $\cond$.
%\[
%\infer[(\cond)]{
%  {x_1} : \bfColor{red}{\N},{x_2} : \bfColor{blue}{\N}
%  \prove \cond(f,g) : \bfColor{oldgold}{\N} \rightarrow \N
%}{
%  x_1 : \bfColor{red}{\N},{x_2} : \bfColor{blue}{\N} \prove f : \N
%  &
%  x_1:\bfColor{red}{\N}, x_2:\bfColor{blue}{\N} \prove g:\bfColor{oldgold}{\N} \rightarrow \N
%}
%\]
%\end{Eg}
%Remark that the first unnamed argument of $\cond(f,g)$ (colored in \bfColor{oldgold}{old gold}) 
%is in connection the type of the first unnamed one of $g$ in the second premise,
%but it has no connection with the first premise. 
%
%
%\subsection{A formal definition of connection}

%15:19 24/03/2025

The definition of connection requires to define an injection 
\[
\ins:\N,\N \rightarrow\N
\]
We set $\ins(p,x)=x+1$ if $x \ge p+1$, and $\ins(p,x)=x$ if $x\le p$.
The role of $\ins$ is inserting one fresh index $p+1$ for a new argument type higher in the proof
connected to no argument in the conclusion.

\begin{definition}[Connection in a proof of  $\LAMBDA$]
Assume $\vec{A} = A_1, \ldots, A_n$, $\vec{B}=B_1, \ldots, B_m$, $\Gamma = \vec{x}:\vec{A}$,
and $\Gamma \vdash t:\vec{B} \rightarrow \N$.

For each $\N$-argument $k$ in $t$, for $t'=t$ or $t'$  immediate subterm of $t$ 
each $\N$-argument $k'$ in $t'$ we define the relation: ``$k',t'$ the successor of $k,t$". We require:
\begin{enumerate}
\item
if $t$ is obtained by a rule $\weak$ from $t'=t$, described by a map $\phi:\Gamma \rightarrow \Delta$,  
then $k = \phi(k')$.
\item
if $t=f(a)$ is obtained by a rule $\apnotvar$ (i.e., $a$ is not a variable) and $t'=f$ 
then $k' = \ins(n,k)$. If $t'=a$ then $k'=k$ and $k' \le n$.
\item
if $t=f(x_i^{A_i})$ is obtained by a rule $\apvar$ and $t'=f$ 
then $k' = \ins(n,k)$or $k'=n+1$ and $k=i$.
\item
if $t=\cond(f,g)$ and $t'=f$ 
then $k' = \ins(n,k)$. If $t'=g$ then $k'=k$.
\end{enumerate}
We require $k = k'$ in all other cases, 
which are: $\Succ $, $\lambda$. 
\\
No connection is defined for the rules $\var$, $0$.
\end{definition}

Below we include some examples. 
We draw in the same color two $\N$-argument which are in connection. 
In the case of the rule $\apvar$, an argument can be connected to two arguments above it.


\begin{Eg}\label{eg:3}%\rm
Atom connection for $\apvar$.
We assume that $x$ is a variable.
\[
\infer[(\apvar)]{
  x_1 : \bfColor{red}{\N},{x_2} : \bfColor{blue}{\N}, x_3  : \bfColor{oldgold}{\N}
  \prove f(x_3) : \bfColor{orange}{\N} \rightarrow \N
}{
  x_1 : \bfColor{red}{\N},{x_2} : \bfColor{blue}{\N}, x_3  : \bfColor{oldgold}{\N}
  \prove f : \bfColor{oldgold}{\N}, \bfColor{orange}{\N} \rightarrow \N
}
\]
\end{Eg}

Remark that the last variable in the conclusion of the rule
is in connection with the first unnamed argument of $f$ (colored in \bfColor{oldgold}{old gold}) 
in the premise, and with the variable with the same name in the premise: there are
two connections.

\begin{Eg}\label{eg:4}%\rm
Atom connection for  $\lambda$-rule.
We assume that $x$ is  a variable.
\[
\infer[(\ap)]{
  x_1: \bfColor{red}{\N}, x_2: \bfColor{blue}{\N}
  \prove \lambda x.b : \bfColor{oldgold}{\N} \rightarrow \N
}{
  x_1: \bfColor{red}{\N}, x_2: \bfColor{blue}{\N}, x:\bfColor{oldgold}{\N} \prove b : \N
}
\]
\end{Eg}
Remark that the first unnamed argument of $\lambda x.b$ (colored in \bfColor{oldgold}{old gold}) 
in the conclusion is in connection with the last variable type in the premise of the rule.
\\

Summing up, when
we move up in a $\apvar$-rule, the type of last free variable corresponds to the type of the first
unnamend argument.
When  we move up in a $\lambda$-rule it is the other way round. 

The connection on $\N$-arguments in a proof $\Pi:\Gamma\vdash t:A$ defines a graph $\Graph(\Pi)$ 
whose nodes are all pairs $(k,u)$, with $u$ subterm of $t$ and $k$ index of some $\N$-argument of  $u$.
$\Graph(\Pi)$ is finite when $t$ is regular.
A trace represents the movement of an input information through the infinite unfolding of a tree.
Formally, we define a trace as a (finite or infinite) path in the graph $\Graph(\Pi)$.

\begin{definition}[Trace for well-typed terms in $\LAMBDA$]
Assume $\Pi$ is any typing proof of $t \in \WTyped$.
\begin{enumerate}
\item
A path of $\Pi$ is any finite or infinite branch $\pi =(i_0, \ldots, i_n, \ldots)$ of $\Pi$.

\item
Assume $\pi =(t_0, \ldots, t_n, \ldots)$ is a path of $\Pi$, finite or infinite. 
A finite or infinite \emph{trace} $\tau$ of $\pi$ in $\Pi$ is a list 
$\tau =( (k_m,t_m), \ldots, (k_n,t_n), \ldots)$ such that for all $i=m,\ldots, n,\ldots$:
\begin{enumerate}
\item
$k_i$ is an index of some $\N$-argument of $t_i$
\item
if $i+1$ is an index of $\tau$ then $(k_{i+1},t_{i+1})$ is connected with $(k_i, t_i)$ in $\Pi$.
\end{enumerate}

\end{enumerate}
\end{definition}

The starting point of a trace could be different from the starting point of the path to which the trace belongs.

%20:50 15/04/2024
%10:35 24/04/2024
%12:28 30/04/2024