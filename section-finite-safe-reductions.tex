
\section{Terms with the Global Trace Condition are Finite for Safe Reductions}

\label{section-finite-safe-reductions}
In this section we prove that for all $n \in \N$, 
every infinite reduction sequence $\pi$ 
from some term of $\GTC$ includes
only finitely many \quotationMarks{$0$-safe} (safe) reduction steps:
from some point on, all reduction steps are \quotationMarks{unsafe},
that is, they occur in the right-hand-side of some $\cond$.
    
From this, we will prove that in $\GTC$ the reduction sequences made of only \quotationMarks{$n$-safe} reductions from some point on stop,
therefore are finite. Namely, this implies strong normalization for \quotationMarks{$n$-safe} reduction steps:
no matter how we reduce within the $n$-safe level of a term, 
eventually no reductions are left, therefore we obtained some $n$-safe normal 
form.
\\

We introduce the property of being \quotationMarks{finite for $n$-safe reductions}.

\begin{definition}
\label{definition-finite-n-safe-reduction}
Assume $t \in \WTyped$. 
We say that $t$ is \emph{finite for $n$-safe reductions} if and only if 
all infinite reduction paths include only finitely many $n$-safe reductions
\end{definition}

to put otherwise: in all infinite reduction paths, from some point on all reductions
are \emph{not} $n$-safe, they contract redexes occurring in the right-hand-side of
$n$ $\cond$-terms. In the following, we explicitly write $t[x_1,\ldots,x_n]$,
when each free variable in a term $t$ is some $x_i$, 
and, under this notation, we also write $t[a_1,\ldots,a_n]$ for 
$t[a_1/x_1,\ldots,a_n/x_n]$. 

It is enough to consider the case of $0$-safe reductions, 
the general case will follow. 
As we previously said, 
we often abbreviate \quotationMarks{$0$-safe} with \quotationMarks{safe}.
We have to prove that all terms in $\GTC$ are finite for safe reductions. 
For, we define total well-typed terms by induction on the type, as in Tait's normalization proof.

%We recall that $u \in \LAMBDA$ is \emph{finite for safe reduction} 
%if and only if all infinite reduction sequences from $t$ include only finitely many "safe" reduction steps
%(Def. \ref{definition-safe-trunk}).

%18:45 26/03/2025

\begin{definition}[Total well-typed terms of $\LAMBDA$]
\label{definition-total-term}
Let $t \in \WTyped$ and $t : T$.
We define \quotationMarks{$t$ is total of type $T$} by induction on $T$

\begin{enumerate}
\item
Let $T$ be any atomic type: $T = \N$ or $=\alpha$ for some type variable $\alpha$.
Then $t$ is total of type $T$ if and only if $t$ is finite for $0$-safe reductions. 

\item
Let $T$ be any arrow type $A \rightarrow B$.
Then $f$ is total of type $T$ if and only if for all total $a$ of type $A$ we have $f(a)$ total of type $B$.
\end{enumerate}

%A \emph{total assignment} $[\vec{v}/\vec{x}]$ is any assignment of total terms to variables of the same type.
%
% A term $t$ is \emph{total by substitution} if and only if:
% $t[\vec{v}/\vec{x}]$ is total for all total assignments $[\vec{v}/\vec{x}]$
A term $t[\vec{x}]$ with free variables $\vec{x}:\vec{B}$ is \emph{total by substitution} if and only if:
$t[\vec{v}]$ is total for all totals $\vec{v}:\vec{B}$. 

Let $U$ be an atomic type, $t[\vec{x}]:\vec{A}\to U$, and $\vec{x}:\vec{B}$. 
A \emph{total assignment} of $t[\vec{x}]$ is $(\vec{v},\vec{a})$, where 
$\vec{v}:\vec{B}$ and $\vec{a}:\vec{A}$ are total terms.
The result of the assignment is  $t[\vec{a}](\vec{v})$.

\end{definition}

By definition, any total term is well-typed, in particular it has exactly one type. 

We define a well-founded relation predecessor relation on terms finite for safe reductions and of type $\N$. We recall that $\reduces$ denotes one reduction
step (contraction of a single redex which is a subterm), 
$\reduces^{*}$ denotes zero or more reduction steps and 
$\reduces^{+}$ denotes one or more reduction steps.

\begin{definition}[The $\Succ$-order]
Assume $t, u \in \WTyped$ and $t, u :\N$ (possibly open terms).
Then:
$
(u \prec t) \Leftrightarrow (t \reduces^{*} \Succ(u))
$
\end{definition}


\begin{Eg}
If $t = \Succ^2(x)$ then $x \prec \Succ(x) \prec t$. 
If $t = \Succ(t)$ then $t \prec t$ and $\prec$ is \emph{not} well-founded on $t$.
\end{Eg}

We did not prove Church-Rosser yet, therefore in principle two maximal
sequences of $\prec$ from the same term of $\CTlambda$ 
could have a different length. However, we can prove that if $t$ is a term finite 
for safe reductions, then any $\prec$-decreasing sequence from $t$ terminates.


\begin{lemma}[The $\prec$-order]
\label{lemma-prec-order}
Assume $t \in \WTyped$ has type $\N$, and $t$ is finite for safe reductions.

\begin{enumerate}
\item
\label{lemma-prec-order-01}
There is no infinite sequence 
$\sigma: t = t_0 \reduces^{*} \Succ(t_1) \reduces^{*} \Succ^2(t_2) \reduces^{*} \ldots$

\item
\label{lemma-prec-order-02}
$\prec$ is well-founded on total terms of type $\N$.
\end{enumerate}
\end{lemma}

%
%\begin{proof}
%\begin{enumerate}
%\item
%%\label{lemma-prec-order-01}
%Assume that there is such $\sigma$ to show a contradiction. Since $t$ is finite for safe-reductions, 
%$\sigma$ only has finitely many safe-reductions. 
%Thus, from some $k\in\N$ there are no more safe reductions from
%$\Succ^k(t_k)$. This implies that for some $\cond$-free term 
%$u$ and some terms $f_1, g_1, \ldots, f_m, g_m$ we have
%$t_k = u[\cond(f_1,g_1), \ldots, \cond(f_m,g_m)]$ and all reductions from $t_k$ on are inside
%some $g_1, \ldots, g_m$. This means for all $h \in \N$, $h \ge k$ we  have
%$\Succ^{h}(t_{h}) =  u[\cond(f_1,g'_1), \ldots, \cond(f_n,g'_m)]$ for some 
%$g'_1, \ldots, g'_m$. This implies that first $h$ symbols of $u$ are $\Succ$.
%This is a contradiction when $h$ is larger than 
%%%%%%%%%%%%%%%%%%%%%%
%% 12:40 14/02/2025 REMOVED: ``$k$, which is''
%%%%%%%%%%%%%%%%%%%%%%
%the number of symbols in $u$.
%
%\item
%%\label{lemma-prec-order-02}
%Assume for contradiction that there is some infinite sequence
%$\ldots t_n \prec \ldots \prec t_2 \prec t_1 \prec t_0$
%from some $t_0:\N$ total. By definition, $t_0$ is finite for safe reductions.
%Then there is some infinite sequence 
%$\sigma: t = u_0 \reduces^{*} \Succ(u_1) \reduces^{*} \Succ^2(u_2) \reduces^{*} \ldots$,
%contradicting point \ref{lemma-prec-order-01} above.
%\end{enumerate}
%\end{proof}


We will prove that if $t$ is not total, then we can assign total terms to  the 
sub-terms of $t$ in any infinite path $\pi$ of a proof $\Pi : \Gamma \vdash t: A$ 
in a way compatible with traces and progress points. In particular, the 
total terms associated to any infinitely progressing trace $\tau$
in $\pi$ will form an infinite decreasing sequence for $\prec$.
This cannot be, therefore non-total terms cannot exists. 


\begin{definition}(Trace-compatible Assignment)
\label{definition-trace-compatible}
Assume $\pi  = (\Gamma_1 \vdash t_1:A_1, \ldots, \Gamma_n \vdash t_n:A_n, \ldots)$ 
is any finite or infinite branch of a typing proof $\Pi$
and $\vec{v} = (r_1, \ldots, r_n, \ldots)$ 
is any sequence of total assignments, where, for each $i$, $r_i=(r_{i,1},\ldots,r_{i,m_i})$ is a sequence of total terms and total assignment for $t_i$. 
$\vec{v}$ is \emph{trace-compatible} in an index $i$ of $\pi$  
if and only if it satisfies the following condition:

  for all $j$  index of an $\N$-argument of $t_i$, 
  all $k$ index of an $\N$-argument of $t_{i+1}$, 
  if $j$ is connected to $k$ then:
 \begin{enumerate}
 \item
 if $j$ progresses to $k$ then $r_{i+1,k} \prec r_{i,j}$ 
 \item
 if $j$ does not progress to $k$ then $r_{i+1,k} = r_{i,j}$.
 \end{enumerate}
$\vec{v}$ is \emph{trace-compatible} if it is trace-compatible in all $i$.
\end{definition}

Now we will prove that if an infinite branch $\pi$ of a proof tree $\Pi:\Gamma \vdash t:A$ 
has a trace-compatible assignment made of total terms, 
then all traces $\sigma$ of $\pi$ progress only finitely many times and the term $t$ is not in $\GTC$
($t$ does not satisfy the global trace condition).



\begin{proposition}[Trace assignment]
\label{prop:trace_assign-finiteness}
Assume $\Pi:\Gamma \vdash t:A$ and there is some infinite path $\pi$ of $\Pi$ for which we have
some \emph{total} trace-compatible assignment  $\rho$ to $\pi$. 
\begin{enumerate}
\item
\label{prop:trace_assign-finiteness1}
Any trace $\sigma$ in $\pi$ progresses only finitely many times.
\item
\label{prop:trace_assign-finiteness2}
$t \not \in \GTC$.
\end{enumerate}
\end{proposition}


%
%\begin{proof}
%\begin{enumerate}
%\item
%%\label{prop:trace_assign-finiteness1}
%By definition of trace-compatible assignment, if at step $i \in \N$ the trace $\sigma$ progresses, 
%then $\sigma(i+1)\prec \sigma(i)$, and if $\sigma$ does not progress, 
%then $\sigma(i+1) = \sigma(i)$
%The assignment is made of total terms, therefore
% $\prec$ is well-founded by lemma \ref{lemma-prec-order}.\ref{lemma-prec-order-02}.
%Thus, any  trace in $\pi$ progresses only finitely many times, as we wished to show.
%
%\item
%%\label{prop:trace_assign-finiteness2}
%By point \ref{prop:trace_assign-finiteness1} above, no trace $\sigma$ 
%from any argument in any term of the branch $\pi$ of $\Tree(t)$ progresses infinitely many times.
%We assumed that $\pi$ is an infinite path in $\Pi$.
%By definition of $\GTC$, we conclude that $t \not \in \GTC$. 
%\end{enumerate}
%\end{proof}
%
%%16:46 04/09/2024

We now continue with our Tait's-style proof of Strong Normalization.
We check that total terms are closed by reductions, by application and by variables.
Any total term is finite for safe reductions.



\begin{lemma}\label{lem:total_value-finiteness}
Assume $t,u,f,a \in \WTyped$, $n \in \N$ and $A,B,T$ are types.

  \begin{enumerate}
  \item
\label{lem:total_value-finiteness1}
    Let $t:A$ and $t \reduces^{*} u$.
    If $t$ is total, then $u$ is total.

  \item
\label{lem:total_value-finiteness2}
    If $f:A \rightarrow B$ and $a:A$ are total terms, then $f(a)$ is total.

  \item
\label{lem:total_value-finiteness2bis}
    $x^T:T$ is total. If $t:T$ is total, then $t$ is  finite for safe reductions.

  \item
\label{lem:total_value-finiteness3}
    Let $U$ be any atomic type, and $t[\vec{x}]:\vec{A}\rightarrow U$ be a term
    whose all free variables are $\vec{x}:\vec{B}$.

    If for all total $\vec{u}:\vec{B}$, $\vec{a}:\vec{A}$ the term 
    $t[\vec{u}]\vec{a}: U$ is \emph{finite for safe reductions}, then
    the term $t[\vec{x}]$ is \emph{total by substitution}.
  \end{enumerate}

\end{lemma}

%
%\begin{proof}
%\begin{enumerate}
%
%\item
%%\label{lem:total_value-finiteness1}
%  We show \emph{point \ref{lem:total_value-finiteness1}}  by induction on $A$. 
%  We assume that $t:A$ and $t \reduces ^*u$.
%    and $t$ is total, in order to prove that $u$ is total.
% By the subject reduction property, $u$ has type $A$.
%\begin{enumerate}
%\item
%  We show the \emph{base case}, namely when $A =\N,\alpha$ is an atomic type.
%  By the assumption, $t$ is total.
%  By definition of $t$ total for $T$ atomic, all infinite 
%  reductions from $t$ only include finitely many safe reductions, for all $n \in \N$.
%  In particular, all infinite reductions $\sigma: t \reduces \ldots \reduces 
%  u \reduces u_1 \reduces u_2 \ldots$ 
%  passing through $u$ only include finitely many safe reductions. We conclude that
%  all infinite reductions 
%  $\sigma': u \reduces u_1 \reduces u_2 \ldots$  from $u$
%  only include finitely many safe reductions. From $u:A =\N,\alpha$ we conclude that  $u$ is total.
%\item
%  We show the \emph{induction case}, namely when $A = (A_1\rightarrow A_2)$.
%  Take any arbitrary total term $a:A_1$ in order to prove that $u(a):A_2$ is total. 
%  Then we have $t(a) \reduces^{*} u(a)$ and 
%  $t(a):A_2$ is total by the assumption that $t$ is total.
%  Hence $u(a)$ is total by $t(a) \reduces^{*} u(a)$ and the induction hypothesis on $A_2$. 
%  We conclude that $u:A_1\rightarrow A_2$ is total. 
%\end{enumerate}
%
%  \item
%%\label{lem:total_value-finiteness2}
%If $f:A \rightarrow B$, $a:A$ are total  terms, then $f(a)$  is total by definition of total.
%
%\item
%%\label{lem:total_value-finiteness3}
%We prove that $x^T:T$ is total. 
%We actually prove a little more, 
%that for all total $\vec{a}:\vec{A}$, if $T = \vec{A} \rightarrow U$ then $x(\vec{a}):U$
%is total. The thesis follows if we take $\vec{a} = \nil$. We argue by induction on $U$. 
%
%\begin{enumerate}
%\item
%Assume $U$ is atomic. Then every reduction on $x(\vec{a}):U$
%takes place in $\vec{a}$. By definition of total
%for an atomic type we have to prove that in all infinite reduction sequences from $x(\vec{a}):U$ 
%there are only finitely many safe reduction. All reductions on $x(\vec{a})$ take place on $\vec{a}$,
%and since each $a_i$ in $\vec{a}$ is total only finitely many safe reductions are possible, as we wished.
%\item
%Assume $U = A_1 \to A_2$. By definition of total
%for an arrow type we have to prove that for all total $a$ we have  $x(\vec{a},a):A_2$ total.
%This follows by induction hypothesis on $A_2$.
%\end{enumerate}
%
%\item
%%\label{lem:total_value-finiteness4}
%\emph{We assume that  $t:U$ is total in order to prove that $t$ is finite for safe reductions},
%for all $n \in \N$.
%We argue by induction on $U$.
%\begin{enumerate}
%\item
%If $U$ is atomic then the thesis is true by definition of total.
%\item
%Suppose $U = (A_1 \rightarrow A_2)$. By point \ref{lem:total_value-finiteness3} above, 
%$x^{A_1}:A_1$ is total, therefore $t(x):A_2$ is total and by induction hypothesis on $A_2$
%any infinite reduction sequence from $t(x)$ only includes finitely many safe reductions. 
%Any infinite reduction sequence 
%$\sigma: t = t_0 \reduces_1 t_1 \reduces_1 t_2 \reduces_1 \ldots$  from $t$ 
%can be raised to an infinite reduction sequence 
%$\tau: t(x) = t_0(x) \reduces_1 t_1(x) \reduces_1 t_2(x) \reduces_1 \ldots$ from $t(x)$
%while preserving the fact that a reduction is safe, because $t$ occurs in no $\cond$ in $t(x)$.
%We conclude that $\sigma$ only includes finitely many safe reductions. 
%\end{enumerate}
%
%\item  
%%\label{lem:total_value-finiteness5}
%We show that $t[\vec{u}]$ is total by induction on the number $|\vec{A}|$ of
%elements of $\vec{A}$.
%\begin{enumerate}
%\item
%  The \emph{base case} $|\vec{A}| = 0$ is immediately shown by the assumption.
%\item
%  We show the \emph{induction case}. Let $\vec{A} = A_0,\vec{A'}$.
%  Take arbitrary totals $\vec{u}:\vec{B}$, $\vec{a'}:\vec{A'}$, and $a_0:A_0$. 
%  By the assumption, we have that $t[\vec{u}]a_0\vec{a'}: U$ is total  for all 
%  vector of total terms $\vec{a'}$. 
%  Then $t[\vec{u}]a_0:\vec{A'}\rightarrow U$ is total for all total $a_0:A_0$
%  by the induction hypothesis on $\vec{A'}\rightarrow U$.
%  By definition of total we deduce that $t[\vec{u}] : A_0,\vec{A'}\rightarrow U$ is total.
%\end{enumerate}
%We conclude that $t[\vec{x}]$ is total by substitution, as we wished to show.
%
%\end{enumerate}
%\end{proof}
%
%%10:28 19/04/2024
%%17:32 24/04/2024

Let $\Pi=(T,\phi)$ be a proof of $\Gamma\vdash t:A$ and $l_n$ be a node of $\Pi$, that is, a list  
of integers $l_n=(e_1, \ldots, e_{n-1}) \in \universe{\Pi}$ for some $n \in \N$.
We write $\Gamma_{n}\vdash t_{n}:A_{n} = \Label(\Pi,l_n)$ for the sequent
 labelling the node $l_n$. We want to prove that all terms of $\GTC$ are total by substitution.
If we consider the substitution of a variable with itself 
(a variable is a total term by 
\ref{lem:total_value-finiteness}.\ref{lem:total_value-finiteness2bis}),
we will deduce that all terms of $\GTC$ are total, 
hence finite for safe reductions, hence strongly normalizing for $0$-safe reductions.
\\

One problem in the proof of the theorem 
is that reduction does not commute with the second argument of substitution. 
That is, if $b \reduces^{*} a$ we cannot deduce that $v[b] \reduces^{*} v[a]$. 
The reason is that we could have infinitely many free occurrences of 
$x$ in $v$, and it could take
an infinite number of steps to reduce to $a$ 
each $b$ which has been replaced to $x$ in $v$.

However, we can prove a weaker property: 
if there is some infinite reduction from $v[a]$, 
then there is some infinite reduction from $v[b]$,
such that if the first reduction has infinitely many safe reduction steps,
then the second reduction has infinitely many safe reduction steps, too.

To this aim, we first define a simulation relation from reductions
from $v[a]$ to reductions from $v[b]$. 

We recall that  $\reduces$, $\reduces^{*}$, $\reduces^{+}$ denote
respectively one, zero or more, one or more reduction steps.


\begin{lemma}[Safety-preserving Simulation]
 \label{lemma-safety-preserving-simulation}
Define a binary relation $R \subseteq \LAMBDA \times \LAMBDA$ by:
$t R u$ if and only if there are $v,a,b \in \LAMBDA$ and a variable $x$
such that:
\begin{center}
$(b \reduces^{*} a)$
  \ \ \  and  \ \ \ 
$(t = v[a/x])$
  \ \ \  and  \ \ \ 
$(u = v[b/x])$
\end{center}

Then $ R$ 
is a \emph{safety-preserving} simulation  on $\LAMBDA$
between $\reduces  $ and $\reduces^{+}$, namely:
\begin{enumerate}
\item
whenever $t,u,t' \in \LAMBDA$, $t R u$ and $t \reduces   t'$ 
then for some $u' \in \LAMBDA$ we have $t' R u'$ and $u \reduces^{+} u'$.
\item
Besides, if $t \reduces   t'$ is safe then some step in
$u \reduces^{+} u'$ is safe, too.
\end{enumerate}
\end{lemma}

%\begin{proof}
%%%%%%%%%%%%%%%%%%%%
%%14:35 14/12/2024
%% ANCORA DA RICONTROLLARE
%%%%%%%%%%%%%%%%%%%%
%Assume that $t \in \LAMBDA$, $u \in \LAMBDA$, $t' \in \LAMBDA$
%and $t R u$ and $t \reduces   t'$. 
%We have to prove that for some $u' \in \LAMBDA$ 
%we have $t' R u'$ and $u \reduces^{+} u'$ and besides that
%if $t \reduces   t'$ is safe then some step in
%$u \reduces^{+} u'$ is safe, too.
%\\
%
%The assumption $t R u$ unfolds to: for some $a, b,w  \subseteq \LAMBDA$,
%some variable $y$, 
%we have $(b \reduces^{+} a)$ and $(t = w[a/y])$ and $(u = w[b/y])$.
%By renaming $y$ in $v$ with some variable $x \not \in \FV(w,a,b)$ 
%we have that $(t = w[x/y][a/x])$ and $(u = w[x/y][b/x])$ and
%$x \not \in \FV(w,a,b)$. If we set $v=w[x/y]$ we have
%$(t = v[a/x])$ and $(u = v[b/x])$ and
%$x \not \in \FV(a,b)$.
%\\
%
%The assumption $t \reduces   t'$ implies that for some unique redex 
%$r \sqsubseteq t$, for some context $C[\cdot]$
%we have $t = C[r]$ and $t' = C[r']$ and $r \reduces r'$. 
%The possible shapes of $r$
%are $r = (\lambda x.c)(d), \cond(f,g)(0), \cond(f,g)(\Succ(e))$ 
%and the corresponding shapes of $r'$ are  $r' = c[d/x], f, g(e)$.
%Thus, for some $r_1, r_2$ we have $r = r_1(r_2)$. 
%%We call $r_1 = (\lambda x.c), \cond(c,d), \cond(c,d)$ the body of $r$ and
%%$r_2= d,0,\Succ(e)$ the argument of $r$.
%Assume $\pi$ is the position of $r$ in $t$. If $r \reduces r'$ is safe, then 
%$\pi$ crosses no right-hand side of any $\cond$.
%We argue by cases on $\pi$.
%Either $\pi$ is some position of $v$ or it is not.
%
%
%\begin{enumerate}
%
%\item
%\emph{Assume that $\pi$ is \emph{not} the position of a node of $v$}. 
%Then there is some free occurrence of $x$ in $v$, with position $\theta$, 
%such that $\pi \ge \theta$. Assume $z$ is the term obtained by replacing this
%occurrence of $x$ with a single variable $x_0 \not \in \FV(v)$.
%Then $v=z[x/x_0]$, therefore $t = z[x/x_0][a/x]$
%and $u = z[x/x_0][a/x]$. We deduce
%$ t  = z[a/x_0,a/x] = $ (by $x \not \in \FV(a,b)$ ) $=z[a/x_0][a/x]$
%and 
%$ u  = z[b/x_0,b/x] = $ (by $x \not \in \FV(a,b)$ ) $=z[b/x_0][b/x]$.
%Let us abbreviate $D[\cdot]=z[\cdot/x_0]$, with $D$ context:
%then $ t  = D[a][a/x]$ and $ut  = D[b][b/x]$.
%For some context $C$ we have $a = C[r] \reduces C[r']$.
%From $x \not \in \FV(a) \cup \FV(b)$ we deduce that there is some context
%$D$ such that: 
%$ t  = D[a][a/x] = D[C[r]][a/x]$ and 
%$ u = D[b][b/x]$ and
%$ t' = D[C[r']][a/x]$. 
%We choose $u' = D[C[r']][b/x]$.
%We deduce $t' R u'$.  
%From $r \reduces   r'$ 
%we deduce $D[r] \reduces  
%D[r']$, then $b \reduces^{*} a = C[r] \reduces  
%C[r']$, hence $b \reduces^{+} C[r']$. 
%Eventually we deduce
%$
%u = D[b][b/x] 
%\reduces^{*} 
%D[C[r]][b/x] 
%\reduces
%D[C[r']][b/x] 
%= u'$.
%If the reduction $t = D[C[r]][a/x]  \reduces D[C[r']][a/x] $ is safe,
%then the last reduction$D[C[r]][b/x] \reduces D[C[r']][b/x] $ in 
%$u \reduces u'$ is safe.
%
%\item
%\emph{Assume that $\pi$ the position of some node $s$ in 
%$v$ and \emph{$s$ is some redex}}.  
%There exists some context $D$ such that $v = D[s]$. Then $r=s[a/x]$
%and $r'=s'[a/x]$ for some redex $s=s_1(s_2)$ of $v$ contracted to $s'$
%with the same reduction used for $r$.
%We deduce that:
%$ t  = v[a/x] = D[s][a/x]$ and 
%$ u = v[b/x] = D[s][b/x]$ and
%$ t' = D[s'][a/x]$. 
%We choose $u' = D[s'][b/x]$. 
%From $b \reduces^{*} a$ we deduce that
%$t' = D[s'][a/x]$ and 
%$u' = D[s'][b/x]$ are related by $R$.
%Thus, we have $t' R u'$. From  $s \reduces   s'$ we deduce that 
%$
%u = D[s][b/x]
%\reduces   
%D[s'][vec{b}/x] = u'
%$.
%If $r \reduces r'$ is safe, then 
%$\pi$ crosses no right-hand side of any $\cond$, therefore the reduction
%$s \reduces s'$ is safe in $u$.
%
%
%\item
%\emph{Assume that $\pi$ the position of some node $s$ in 
%$v$,  yet \emph{$s$ is no redex}}.  
%There exists some context $D$ such that $v = D[s]$. 
%$\pi$ is the position of an application $r_1(r_2)$ in $t$, therefore
%$\pi$ is the position of some application $s = s_1(s_2)$ is $v$.
%If $s$ is no redex, then either $s_1, r_1$ do not start with the same symbol,
%or and $s_2, r_2$ do not start with the same symbol. In the first sub-case
%we have $s_1=x$, $r_1 = s_1[a/x] = a$ and $r = r_1(r_2) = a(r_2)$.
%In the second sub-case
%we have $s_2=x$, $r_2 = s_2[a/x] = a$ and $r = r_1(r_2) = r_1(a)$.
%We cannot have both case at the same time, 
%because the type of $r_2$ is the type of an argument of $r_1$. 
%Therefore $r_1$, $r_2$ have two different types and $a$ cannot have both types.
%Thus, if $s_1=x$ then $r_2,s_2$ start with the same symbol, and if
%$s_2=x$ then $r_1,s_1$ start with the same symbol.
%
%\begin{enumerate}
%\item
%\emph{Sub-case: $s_1=x$ and $r_2,s_2$ start with the same symbol}. 
%We have $r1(r_2) = a(s_2[a/x]) = $ (by $x \not \in \FV(a)$) $ a(s_2)[a/x]$ 
%If we re-define $s_1=a$ we have that $s_1$ has the same starting symbol
%as $r_1$ and $s_2$ has the same starting symbol as $r_2$.
%Thus, $s = s_1(s_2)$ is a redex, it reduces to some $s'$ with the same reduction
%we have in $r \reduces r'$, and by unicity of contractum we have $r' = s'[a/x]$.
%For the context $D$ such that $v=D[s]$
%we have: 
%$ t  = v[a/x] = D[s][a/x] = D[a(s_2)][a/x]$ and 
%$ u = D[b(s_2)][b/x] $ and
%$ t' = D[s'][a/x]$. 
%We choose $u' = D[s'][b/x]$ 
%and we deduce $t' R u'$. 
%From $b(s_2) \reduces^{*} a(s_2) = s \reduces   s'$
%we deduce that $u = D[b(s_2)][b/x] \reduces^{*} 
%D[s][b/x] \reduces D[s'][b/x] = u'$.
%
%\item
%\emph{Sub-case: $s_2=x$ and $r_1,s_1$ start with the same symbol}. 
%We have $r = r_1(r_2) = r_1(a)$ and
%$r_1 = s_1[a/x]$ for some $s_1$ with the same starting symbol as $r_1$.
%We re-define $s_2=a$. Then $s_2=a$ has the same starting symbol as $r_2=a$,
%therefore $ s = s_1(s_2) = s_1(a)$ reduces to some $s'$ such that $r' = s'[a/x]$
%For some context $D$ we have: 
%$ t  = D[s_1(a)][a/x]$ and 
%$ u = D[s_1(b)][b/x]$ and
%$ t' = D[s'][a/x]$. 
%We choose $u' = D[s'][b/x]$ 
%and we deduce $t' R u'$. 
%From $s_1(b) \reduces^{*} s_1(a) = s \reduces   s'$
%we deduce that $u = D[s_1(b)][b/x] \reduces^{*} D[s_1(a)][b/x] 
%= D[s_1(s_2)][b/x] 
%\reduces D[s'][b/x] = u'$.
%\end{enumerate}
%
%In all cases, if $t \reduces   t'$ is safe then the reduction $s[a/x] \reduces   s'[a/x]$ 
%is safe in $D[s][a/x] \reduces  D[s'][a/x] = t'$
%and therefore the reduction $s[b/x] \reduces   s'[b/x]$
%is the last reduction step in  $u \reduces   D[s'][b/x] = u'$ and it is
%safe, too.
%\end{enumerate}
%\end{proof}

Now we can prove the required property for reductions with infinitely
many safe reduction steps.

\begin{lemma}[Safe infinite reductions and Substitution]
 \label{lemma-safe-infinite-substitution}
Let $a \in \LAMBDA$, 
$b \in \LAMBDA$ 
and $v \in \LAMBDA$. 
If $b \reduces^{+} a$ and there is some reduction with 
\emph{with infinitely many safe steps} from $v[a/x]$, 
then there is some reduction with
\emph{with infinitely many safe steps} from $v[b/x]$.
\end{lemma}

%19:18 14/12/2024
%%%%%%%%%%%%%%%%%%%%%%%%
% FIXED 13:10 14/12/2024
%%%%%%%%%%%%%%%%%%%%%%%%
%
%\begin{lemma}[Safe-infinite reductions and Substitution]
% \label{lemma-safe-infinite-substitution}
% Assume $a \reduces^{*} a'$ and there is some safe-infinite reduction $\sigma$ from $v[a']$.
% Then there is some safe-infinite reduction from $v[a]$.
%\end{lemma}
%
%\begin{proof}
%We prove that if $v[a'] \reduces w$ then for some $z$ we have $w=z[a']$ and $v[a] \reduces^+ z[a]$,
%and if $v[a'] \reduces z[a']$ is a safe then the last reduction in $v[a] \reduces^+ z[a]$ is safe.
%The proof is by cases on the reduction $v[a'] \reduces w$.  We argue by cases.
%
%\begin{enumerate}
%
%\item
%Assume the reduction $v[a'] \reduces w$ is on a redex of the form $r[a']$, with both the left
%and right hand-side of $r$ not included in any $a'$ obtained by substitution. 
%In this case we have $v[a'] = e[r[a'],a'] \reduces e[s'[a'],a']$, and
%$r[a'] = \cond(f[a'],g[a'])(0) \reduces f[a'] = s[a']$ or 
%$r[a'] = \cond(f[a'],g[a'])(S(t[a'])) \reduces g[a'](t[a']) = s[a']$
%or $r[a'] = (\lambda x.t[a',x])(u[a']) \reduces t[a',u[a']] = s[a']$. 
%We define $z[a'] = e[s'[a'],a']$. 
%In all three sub-cases we have $r[a] \reduces s[a]$ and therefore $v[a] \reduces e[r[a],a] = z[a]$
%if $v[a'] \reduces z[a']$ is safe then $v[a] \reduces z[a]$ is safe.
%
%\item
%  {\Large\color{red} This proof should be fixed}
%Assume the reduction $v[a'] \reduces w$ is on a redex of the form $r$, with either the left
%or right hand-side of $r$ not included in some $a'$. In this case either $r$ is included in some $a'$ 
%obtained by substitution, or the left or right hand side of $r$ is equal to some $a'$ obtained by substitution.
%We cannot have both the left and right hand side equal to $a$, because the two sides of any redex 
%have a different type.
%
%In this case there is a single $a'$ whose value matters, that is, $v[a] = v_0[a][a]$, with the first $a$ denoting
%the single occurrence of $a$ we are speaking about. We choose $v'[.] = v_0[a'][.]$, then we have
%$v'[a']=v_0[a'][a']=v[a']$ by definition, and $v[a] = v_0[a][a] \reduces^{*} v_0[a'][a] = v'[a]$ 
%because $a \reduces^{*} a'$ and there is a single occurrence of $a$ in $v_0[a][.]$. 
%
%The reduction on $v'[a']$ has both the left
%and right hand-side of $r$ not included in any $a'$ obtained by substitution, because this unique
%$a'$ is not part of $v'[.]$. We conclude from the previous case applied to $v'[a]$, $v'[a']$.
%
%\end{enumerate}
%Then by induction on $n' \in \N$ we prove that if $v[a'] \reduces^{n'} w$ ($n'$ steps) 
%then for some $z$, some $n \ge n'$ we have $w=z[a']$ and $v[a] \reduces^{n} z[a]$,
%with the number of safe reductions in  $v[a'] \reduces^{n'} z[a']$ less or equal than the number
%of safe reductions in $v[a] \reduces^{n} z[a]$.
%
%\end{proof}
%%%%%%%%%%%%%%%%%%%%%%%%
% FIXED 13:10 14/12/2024
%%%%%%%%%%%%%%%%%%%%%%%%

The Lemma above will be enough to prove our Main Theorem.

%08:13 05/09/2024

\begin{theorem}[Main Theorem]
\label{theorem-main-finite-safe-reduction}
  Assume $\Pi:\Gamma\vdash t : A$ (hence $t\in \WTyped$).
  If $t$ is \emph{not} total by substitution, then $t \not \in \GTC$, i.e.:
  there is some infinite path $\pi = (e_1, e_2, \ldots)$ of $\Pi$ with no infinite progressing trace. 
\end{theorem}

%19:01 16/04/2024
%18:11 30/04/2024
%13:39 04/06/2024

%
%\begin{proof}
%  Assume that $t$ is not total by substitution. 
%  Let $\vec{x}:\vec{D}$ and $\vec{A}\rightarrow U$ for some \emph{atomic} $U$ 
% be $\Gamma$ and $A$, respectively.
%  By Proposition \ref{prop:trace_assign-finiteness}.\ref{prop:trace_assign-finiteness2} it is enough to prove that
%  $\Pi$ has some infinite branch $\pi=(e_1, e_2, \ldots)$ 
%  and some \emph{total} trace-compatible assignment $\rho$ for $\pi$.
%  By case \ref{lem:total_value-finiteness3} of Lemma~\ref{lem:total_value-finiteness},
%  there exist total terms $\vec{a}:\vec{A}$ and $\vec{d}:\vec{D}$ for which there is
%  an infinite reduction $\sigma$ from $t[\vec{d}]\vec{a}$ having infinitely many safe reductions.
%  By induction on $i \in \Nat$, for each $i$, we construct a path 
%  $l_i = (e_1,\ldots,e_{i-1}) \in \universe{\Pi}$
%  and a total assignment $\vec{v_i} = (\vec{d_i},\vec{a_i})$ such that
%  $(\vec{v_1},\ldots,\vec{v_i})$ is a total trace-compatible assignment for the node $l_i$. 
%
%%  We have to find some $(l_i,\vec{d_i},\vec{a_i})$ such that:
%%  \begin{itemize}
%%  \item[(i)]
%%%15:38 19/04/2024
%%   $l_i = (e_1, \ldots, e_{i-1}) \in \universe{\Pi}$ 
%%     %and $\Label(\Pi,l_i) = \vec{x_{i}}:\vec{D_{i}}\vdash t_{i} : \vec{A_{i}}\rightarrow\N$; 
%%%  \item[(ii)]
%%%    $t_{i}$ is not total by substitution;
%%  \item[(ii)]
%%    $\vec{d_{i}}:\vec{D_i}$ and $\vec{a_i}:\vec{A_i}$ are total terms
%%    such that $t_{i}[\vec{d_i}]\vec{a_i}$ is not total;
%%  \item[(iii)]
%%    the total assignment 
%%    $\vec{d_1},\vec{a_1}$, \ldots, $\vec{d_i},\vec{a_i}$) is trace-compatible for $l_i$.
%%    %\Daisuke{mynote:write this more clearly}
%%  \end{itemize}
%  
% We recall that \quotationMarks{trace compatible in $i$} means: 
% if $i-1$ is a progress point, namely if $t_{i-1}=\cond(f,g)$ and $e_i=2$ 
%    and $t_i=g$,
%    then $\vec{a_{i-1}} = a',\vec{a'}$ 
%    and $a'$, the first unnamed argument of $\cond(f,g):\N \rightarrow A$, reduces to $\Succ(a'')$
%     while $\vec{a_i} = a'',\vec{a'}$.
%  In all other cases two corresponding arguments are equal.
%
%  We first define $(l_1,\vec{d_1},\vec{a_1})$ for the root node $t$ of $\Pi$.
%  In this case $l_1 = \nil$ 
%   and $(\vec{d},\vec{a})$ are total terms such that $t[\vec{d}](\vec{a})$ is not total.
%  Trace compatibility is vacuous because the branch $l_1$ does not contain two nodes.
%  %Points (i), (ii), (iii), (iv) are immediate.
%
%  Next, assume that $(l_i,\vec{d_i},\vec{a_i})$ is already constructed.
%  Then we define $(l_{i+1},\vec{d_{i+1}},\vec{a_{i+1}})$ by the case analysis on
%  the last rule for the node $l_i$ in $\Pi$. 
%
%%%%%%%%%%%%%%%%%%%%%%%%%%%%%%%%%
%% APPARENTLY THE CASE $\struct(f)$ CAN BE SIMPLIFIED TO WEAKENING-Stefano
%%%%%%%%%%%%%%%%%%%%%%%%%%%%%%%%%
%
%
%
%\begin{enumerate}
%
%\item
%%WEAK
%  The case of $\weak$, namely
%  $\Label(\Pi,l_{i+1}) = \Gamma \vdash t_{i+1}:\vec{A_i}\rightarrow U$
%  is obtained from the induction hypothesis for
%  $\Gamma' \vdash t_{i}:\vec{A_i}\rightarrow U$. We have:
%
%\begin{enumerate}
%\item
% $t_i = t_{i+1}$. 
%\item
%  $\Gamma = x_1:D_1,\ldots,x_n:D_n$, $\Gamma' = x'_1:D'_1,\ldots,x'_m:D'_m$, and $\Gamma \subseteqsim \Gamma'$
%  with an injection $\phi:\{1,\ldots,n\}\to\{1,\ldots,m\}$ between contexts. 
%\item
%  $x_i = x'_{\phi(i)}$ and $D_i = D'_{\phi(i)}$ for all $i \in \{1,\ldots,n\}$.
%\end{enumerate}
%
%  By the induction hypothesis, 
%  $t_i[d_{i,\phi(1)}/x_1,\ldots,d_{i,\phi(n)}/x_n]\vec{a_i} 
%   = t_i[d_{i,1}/x'_1,\ldots,d_{i,m}/x'_m]\vec{a_i}$ is not total,
%  where $\vec{d_i} = d_{i,1}\ldots d_{i,m}$.
%  Then we define $l_{i+1} = l_i\conc(1)$ taking the unique child node of $l_i$ in $\Pi$, and we
%  also define $\vec{d_{i+1}}$ and $\vec{a_{i+1}}$ by $d_{i,\phi(1)}\ldots d_{i,\phi(n)}$
%  and $\vec{a_i}$, respectively. This is an assignment with total terms.
%  $((\vec{d_i},\vec{a_i}),(\vec{d_{i+1}},\vec{a_{i+1}}))$
%  is trace compatible for $(t_i,t_{i+1})$: if two arguments are connected then they are assigned
%  to the same term. 
%
%\item
%%VAR
%  The case of $\var$-rule, namely $\Label(\Pi,l_i) = \Gamma\vdash x:D$ 
%  for some $x_{i,k}:D_{i,k} = x:D \in \Gamma$
%  cannot be, because $t_i [\vec{d_i}/\vec{x_i}] = d_{i,k}$ is total  by assumption on $\vec{d}$.
%  
%\item
%%0-RULE
%  The case of $0$-rule, namely $\Label(\Pi,l_i) = \Gamma\vdash 0:\N$, 
%  cannot be. Indeed, $t_i = 0$ is total because $0$ is a numeral.
%
%%12:58 06/06/2024
%
%\item 
%%SUCC 
%  The case of $\Succ$-rule, 
%namely $\Label(\Pi,l_i) = \Gamma\vdash t_i: \N = \Gamma\vdash \Succ(u): \N$
%  for some $u$ is obtained from our assumptions on
%  $\Gamma\vdash u: \N$. In this case $\vec{a_i}$ is empty, $t_{i+1}=u$, and
%  by the induction hypothesis $\Succ(u)[\vec{d_i}]:\N$ has an infinite reduction with
%  infinitely many safe reductions
%  $\Succ(u) \reduces  \Succ(u_1) \reduces \Succ(u_2) \reduces \ldots$,
%  all taking place on $u$.
%  Then, by the definition of total, $u[\vec{d_i}] =t_{i+1}[\vec{d_i}] $ is not total, because the
% $u \reduces_1  u_1 \reduces_1 u_2 \reduces_1 \ldots$ is an infinite reduction with
%  infinitely many safe reductions.
%
%  We define $e_{i}=1$, 
%  the index of the unique child node of $l_{i+1}=l_i\conc(1)$ in the $\Succ$-rule, and
%  we also define $\vec{d_{i+1}} = \vec{d_i}$ and $\vec{a_{i+1}} = ()$. 
%  This is an assignment with total terms and 
%  trace compatible for $(t_i,t_{i+1})$: if two arguments are connected then they are assigned
%  to the same term. 
%
%%13:05 06/06/2024
%
%
%\item
%  The case of the $\apnotvar$-rule, namely 
%  $\Label(\Pi,l_i) = \Gamma\vdash t_i: \vec{A}\rightarrow U$, 
%  with $t_i = f[\vec{x}](u[\vec{x}])$ for some $f$ and $u$.
%  The premises of the $\apnotvar$-rule
%   are $\Gamma\vdash f[\vec{x}]: B \rightarrow \vec{A}\rightarrow U$ 
%  and $\Gamma\vdash u[\vec{x}]: B$, where $u$ is \underline{not} a variable.
%  By the induction hypothesis, there is some infinite reduction from
%  $t_i[\vec{d_i}]\vec{a_i} = f[\vec{d_i}](u[\vec{d_i}])\vec{a_i}:U$ with infinitely many safe-reductions.
%  We argue by cases on the statement: \emph{$u[\vec{d_i}]:B$ is total}.
%
%
%%11:04 05/06/2024
%
%
%
%\begin{enumerate}
%\item
%  We first consider the subcase: \emph{$u[\vec{d_i}]:B$ is total}.
%  We define $b = u[\vec{d_i}]$, then $l_{i+1}=l_i\conc(1)$, taking the first premise of the rule,
%  and we define $\vec{d_{i+1}} = \vec{d_i}$ and $\vec{a_{i+1}} = b,\vec{a_i}$. 
%  This is an assignment with total values, and providing an infinite reduction with infinitely many safe
%  reductions, as expected. 
%  The connection from 
%  $(\vec{d_i},\vec{a_i})$ to $(\vec{d_{i+1}},\vec{a_{i+1}}) = (\vec{d_i},b,\vec{a_i})$ is
%  trace-compatible: all connected 
%  $\N$-argument of $t_{i}=f(u)[\vec{x}]$ and $t_{i+1}[\vec{x}] = f[\vec{x}]$ are the same,
%  because the only fresh argument of $f[\vec{x}]$ 
%  is $b$ and no argument of $f(u)[\vec{x}]$ is connected to it.
%\item
%  Next we consider the subcase that \emph{$u[\vec{d_i}]:B$ is not total}.
%  Let $B = \vec{C}\rightarrow U$.
%  By lemma \ref{lem:total_value-finiteness}.\ref{lem:total_value-finiteness3}
%  there is a sequence of values $\vec{c}:\vec{C}$ and an infinite reductions from 
%  $u[\vec{d_i}]\vec{c}: U$ with infinitely many safe reductions.
%  Define $l_{i+1}= l_i \conc (2)$ taking the second premise of the rule,
%  and define $\vec{d_{i+1}} = \vec{d_i}$ and $\vec{a_{i+1}} = \vec{c}$. 
%  This is an assignment with total terms providing an infinite reductions with 
%  infinitely many safe reductions, as expected.
%  The connection from 
%  $(\vec{d_i},\vec{a_i})$ to $(\vec{d_{i+1}},\vec{a_{i+1}}) = (\vec{d_i},\vec{c})$ is
%  trace compatible: all connected $\N$-argument of $t_{i} = f(u)[\vec{x}]$ and $t_{i+1}=u[\vec{x}]$ are 
%  in $\vec{d_i}$ and therefore are the same, and no unnamed arguments in $\vec{a_i}$
%  and $\vec{a_{i+1}}$ are connected each other.
% \end{enumerate}
%
%\item
%  The case of $\apvar$-rule, namely 
%  $\Label(\Pi,l_i) 
%  = 
%  \Gamma \vdash f[\vec{x}](x): \vec{A}\rightarrow U$ is obtained from
%  $\Gamma \vdash f[\vec{x}]: D,\vec{A} \rightarrow U$,
%  where $\Gamma=x_1:D_1,\ldots,x_m:D_m$ and $x:D\in\Gamma$, therefore
%  $x:D = x_k:D_k$ for some $k \in [1,m]$. 
%  By the induction hypothesis, there is an infinite reduction from $f[\vec{d_i}]d_{i,k}\vec{a_i}: U$
%  with infinitely many safe reductions,
%  where $\vec{d_i} = (d_{i,1},\ldots,d_{i,m})$. 
%  Define $l_{i+1}=l_i\conc(1)$ as the unique child of $l_i$ in $\Pi$, and
%  also define $\vec{d_{i+1}} = \vec{d_i}$ and $\vec{a_{i+1}} = d_{i,k},\vec{a_i}$. 
%  This is an assignment with total terms providing an infinite reductions with 
%  infinitely many safe reductions, as expected.
%  The connection from
%  $(\vec{d_i},\vec{a_i})$ to
%  $(\vec{d_{i+1}},\vec{a_{i+1}}) = (\vec{d_i},d_{i,k},\vec{a_i})$
%  is trace compatible: all connected $\N$-arguments in $\vec{d_i},\vec{a_i}$ and 
%  $\vec{d_{i+1}},\vec{a_{i+1}}$ are the same.  
%  The only difference between the two assignments
%  is that, if $D_k = \N$, the value $d_{i,k}$ of type $\N$ for the variable $x_k$
%  in $t_i[\vec{x}]=f[x_1,\ldots,x_k,\ldots,x_m](x_k)$ is duplicated to the term $d_{i,k}$ 
%  assigned to the first unnamed argument of $ f[\vec{x}]$. 
%
%%08:40 05/09/2024
%
%\item
%  The case of $\lambda$-rule, namely when a sequent
%  $\Label(\Pi,l_i) = 
%    \Gamma\vdash t_i : A, \vec{A} \rightarrow U$ with $t_i = \lambda x^A.u[\vec{x},x]$
%  is obtained from
%  $\Gamma,x:A\vdash t_{i+1}[\vec{x},x^A]:\vec{A}\rightarrow U$, 
%  where $t_{i+1}[\vec{x},x]=u[\vec{x},x]$.
%  By the induction hypothesis we have an infinite reduction sequence $\sigma$ from
%  $t_i[\vec{d_i}]\vec{a_i} = (\lambda x.(u[\vec{d_i},x]))\vec{a_i}: U$ with infinitely many safe
%  reductions,
%  where $\vec{a_i} = a,\vec{b}$. 
%  Define $l_{i+1}=l_i\conc(1)$ as the unique child of $l_i$ in $\Pi$,
%  and $\vec{d_{i+1}} = \vec{d_i},a$ and $\vec{a_{i+1}} = \vec{b}$. 
%  This is an assignment with total terms.
%    The connection from 
%  $(\vec{d_i},\vec{a_i})$ to $(\vec{d_{i+1}},\vec{a_{i+1}}) = (\vec{d_i},a, \ \vec{b})$ is
%  trace compatible: all connected $\N$-argument of 
%  $t_{i}=\lambda x^A.u[\vec{x},x]$ and $t_{i+1}=u[\vec{x},x]$ are 
%  the same, except for the first unnamend argument $a$ of $\vec{a_i}$ which is moved to
%  the last named argument of $u[\vec{x},x]$, with name $x$.
%  
%  We have to prove that there is some infinite reduction from 
%  $t_{i+1}[\vec{d_{i+1}}](\vec{a_{i+1}})$ with infinitely
%  many safe reductions. We argue by case.
%
%\begin{enumerate}
%\item
% Suppose all reductions in the infinite reduction $\sigma$ 
%  are inside $\lambda x.(u[\vec{d_i},x])$ or $\vec{a_i}$.
% Then there are finitely many safe reductions in $\vec{a_i}$ because $\vec{a_i}$ is total.
% Thus, there are infinitely many safe reductions on the part of $\sigma$
%  taking place on $\lambda x.(u[\vec{d_i},x])$.
%  We conclude that there is an infinite reduction from $u[\vec{d_i},x]$ with infinitely many safe reduction.
%  This is true for $u[\vec{d_i},a]$, too, because reductions and safe reductions are closed by substitution
%  of $x$ with $a$.
%
%%09:39 05/09/2024
%%non è vero che a-->a' implica v[a] ---> v[a'] ma è vero che se da v[a'] ci sono infinite safe-reductions
%% allora da v[a] ci sono infinite safe-reductions
%\item
% Suppose there is some reduction in $\sigma$ contracting the first $\beta$-redex.
% Then  $(\lambda x.(u[\vec{d_i},x]))a\vec{b}$ reduces first to some
% $ (\lambda x.v[x])a'\vec{b'}$, then to $v[a']\vec{b'}$, with: 
% $u[\vec{d_i},x] \reduces^{*} v[x]$ and $a\reduces^{*} a'$ and $\vec{b} \reduces^{*} \vec{b'}$.
% Then the reduction sequence $\sigma$ continues 
%  with some infinite reduction $\sigma'$ from $v[a']\vec{b'}$, including infinitely many safe reductions. 
% From $u[\vec{d_i},x] \reduces^{*} v[x]$ and $\vec{b} \reduces^{*} \vec{b'}$ we deduce that 
% \[
% t_{i+1}[\vec{d_i},a]\vec{b}  
% \quad =\quad
% u[\vec{d_i},a]\vec{b} 
% \quad\reduces^{*}\quad
% v[a]\vec{b}
% \quad\reduces^{*}\quad
% v[a]\vec{b'} 
% \ \ \ \mbox{(1)}
% \]
% We proved that there is a reduction $\sigma'$ including infinitely many safe reductions from  $v[a']\vec{b'}$.
% From $a \reduces^{*} a'$ and Lemma \ref{lemma-safe-infinite-substitution} 
% we deduce that there is a reduction including infinitely many safe reductions from  $v[a]\vec{b'}$.
% From $(1)$ above we conclude that the same holds for
%   $t_{i+1}[\vec{d_i},a]\vec{b}$.
%\end{enumerate}
%
%%13:55 06/06/2024
%
%
%\item
%  The case of $\cond$-rule, namely a sequent
%  $\Label(\Pi,l_i) = \Gamma\vdash t_i:\N,\vec{A}\rightarrow U$
%  having $t_i[\vec{x}] = \cond(f[\vec{x}],g[\vec{x}])$,
%  and obtained from 
%  $\Gamma\vdash f[\vec{x}]:\vec{A}\rightarrow U$
%  and
%  $\Gamma\vdash g[\vec{x}]:\N,\vec{A}\rightarrow U$. 
%  By the induction hypothesis, there is some infinite reduction sequence $\sigma$ from
%  $t_i[\vec{d_i}]a \vec{b} = \cond(f[\vec{d_i}],g[\vec{d_i}])a\vec{b}: U$
%  with infinitely many safe reductions,
%  where $\vec{a_i} = a,\vec{b}$. 
%  We argue by case.
%
%\begin{enumerate}
%\item
%  \emph{Suppose the reduction sequence $\sigma$ never contracts the leftmost $\cond$.}
%  $a$ and $\vec{b}$ are total, only finitely many reduction on them are possible.
%  Then infinitely many safe reductions take place on $f[\vec{d_i}]$ or $g[\vec{d_i}]$. 
%  All these safe reduction are in $f[\vec{d_i}]$, because
%  a reduction inside $g[\vec{d_i}]$ is in the right-hand-side of some $\cond$, therefore it is not safe.
%  The safe-infinite reduction from $f[\vec{d_i}]$ can be raised to an infinite reduction from 
%  $f[\vec{d_i}](\vec{b})$, with the same safe reductions, therefore with infinite safe reduction.
%  In this case we take $f[\vec{x}],\vec{d_i},\vec{b}$ as next step of our path in $\Pi$:
%  we choose $l_{i+1} = l_i \conc (1)$, $\vec{d_{i+1}} = \vec{d_i}$, and
%  $\vec{a_{i+1}} = \vec{b}$.  This is an assignment with total terms.
%
%  We have trace-compatibility because: 
%  each argument of $t_i$ is connected to some equal argument of $t_{i+1}[\vec{x}]=f[\vec{x}]$,
%  the first argument of $t_i[\vec{x}]$ disappears but it is connected to no $\N$-argument in $f[\vec{x}]$.
%
%\item
%  \emph{Suppose the reduction sequence contracts the leftmost $\cond$-redex at some step.}
%  Then $t_i[\vec{d_i}]a \vec{b} \reduces^{*} \cond(f',g')a''\vec{b'}$, with $a'' = 0$ or $a'' = \Succ(a')$. 
%  Then in the first subcase $\cond(f',g')a''\vec{b'} \reduces f'\vec{b'}$, in the second subcase
%  $\cond(f',g')a''\vec{b'} \reduces g' a' \vec{b'}$, 
%  with $f[\vec{d_i}] \reduces^{*} f'$, $g[\vec{d_i}] \reduces^{*} g'$, and $\vec{b} \reduces^{*} \vec{b'}$.
%  After this $\cond$-reduction we have a reduction sequence with infinitely many safe reductions.
%  We prove our thesis by the subcases on $a''$.
%   
%%19:37 05/09/2024
%
%\begin{enumerate}
%\item
%  \emph{Suppose $a'' = 0$}.
%   In this case we take $f[\vec{x}],\vec{d_i},\vec{b}$ as the next step of our path in $\Pi$:
%  we choose $l_{i+1} = l_i \conc (1)$, $\vec{d_{i+1}} = \vec{d_i}$, and $\vec{a_{i+1}} = \vec{b}$. 
%  This is an assignment with total terms.
%  $f[\vec{d_{i+1}}](\vec{b})$ reduces to $f' \vec{b'}$, then we have infinitely many safe reductions.
%
%  We have trace-compatibility because: 
%    each argument of $t_i$ is connected to some equal argument of $t_{i+1}[\vec{x}]=f[\vec{x}]$,
%    the first argument of $t_i[\vec{x}]$ disappears but it is connected to no $\N$-argument in $f[\vec{x}]$.
%%12:53 05/06/2024
%
% \item
%  \emph{Suppose $a'' = \Succ(a')$}. 
%  We choose $l_{i+1} = l_i \conc (2)$, $\vec{d_{i+1}} = \vec{d_i}$, and
%  $\vec{a_{i+1}} = a',\vec{b}$. This is an assignment with total terms: $a'$ is total because
%  all reduction sequences from $a'$ can be raised from a reduction sequences from $a'' = \Succ(a')$
%  with the same safe-reduction. From $a \reduces^{*} a''$ we deduce that
%  there are finitely many safe-reductions from $a''$, therefore finitely many those from $a'$.
% 
%  We have trace-compatibility because: 
%  each argument of $t_i[\vec{x}]$ is connected to some equal argument of 
%    $t_{i+1}[\vec{x}]=g[\vec{x}]$,
%    but for the first unnamed argument $a$ of $t_i[\vec{x}]$ 
%    which is connected the first unnamed argument of $g[\vec{x}]$.
%    This is fine because in the second premise of a $\cond$ 
%    the trace progresses and we have $a' \prec a$, because $a \reduces^{*}  a'' = \Succ(a')$.
%
%  \end{enumerate}
% \end{enumerate}
%\end{enumerate}
%
%By the above construction, we have an infinite path $\pi = (e_1,e_2,\ldots)$ in $\universe{\Pi}$
%and a trace-compatible assignment, as we wished to show.
%
%%  Since $\Pi$ satisfies the global trace condition, $\vec{e}$ contains a progressing trace
%%  $(k_{a},k_{a+1},\ldots)$, where, for each $a\le i$, $k_i$ is an $\N$-argument of $t_i$, 
%%  and $k_{i+1},t_{i+1}$ is the successor of $k_i,t_i$. 
%%  Let $n_i$ be an numeral in $\vec{d_i},\vec{a_i}$ at index $k_i$ for each $a\le i$.
%%  Then the sequence $(n_a,n_{a+1},\ldots$ decreases at each progressing point.
%%  This means that it decreases infinitely many times
%%  since $(k_{a},k_{a+1},\ldots)$ has infinitely many progressing point.
%%  Finally we have a contradiction. 
%  
%\end{proof}
%
%%14:09 06/06/2024

\subsection{Strong Normalization for safe reductions in $\CTlambda$
and Failure of the Closure under Limit for $\CTlambda$}

From this theorem we derive the strong normalization result 
for safe reductions on terms of $\GTC$
and for all terms which satisfy the global trace condition.

\begin{corollary}\label{cor:SN_GTC}
  Assume  $t \in \CTlambda$ and $n \in \N$.
  \begin {enumerate}
  \item
    $t$ is total and all reduction sequences
    from $t$ include only finitely many $0$-safe reductions.
  \item
    $t$ is total and all reduction sequences
    from $t$ include only finitely many $n$-safe reductions.
  \item
    $t$ strongly $n$-safe-normalizes, and in all infinite reductions from $t$
    from some point on all reductions are \emph{not} $n$-safe.
  \item
    All infinite reduction sequences in $\CTlambda$ from $t$ have some limit in 
    $\LAMBDA$, a limit which is sometimes outside $\CTlambda$.
    %Kimura: {\rm{\color{red}\Large Needs to be checked}}
  \end{enumerate}

\end{corollary}

%{\color{red} Kimura}
The last point of the Corollary says that 
$\CTlambda$ is {\bf \emph{not} closed under infinite reductions}:
the limit of any infinite reduction of any term of $\CTlambda$
exists in $\LAMBDA$, yet it is often not in $\CTlambda$
because it is often not regular. An example is the limit
normal form of the interval map $\interval$.

%%%%%%%%%%%%%%%%%%%%%%%%%%%%%%%%%%%%%
%MODIFIED 11/02/2025 to make precise the infinitely progressing trace
%%%%%%%%%%%%%%%%%%%%%%%%%%%%%%%%%%%%%
%
We include a simpler example $f = \lambda x^\N.\cond(x,f(\Succ x))$
such that the limit normal form of $f(x)$ is not regular. 
We will prove that $f$ is the unary algorithm for addition. 
$f$ is regular. $f$ has a type $\N,\N\rightarrow\N$ and we trace the second 
argument $\bfColor{red}{\N}$ of $f$. The unique infinite branch of $f$ is 
$$
f : \N,\bfColor{red}{\N}\rightarrow\N, \ \ \ 
\cond(x,f(\Succ x)) : \bfColor{red}{\N}\rightarrow\N, \ \ \ 
f(\Succ x) : \bfColor{red}{\N}\rightarrow\N, \ \ \ 
f : \N,\bfColor{red}{\N}\rightarrow\N, \ \ \ 
\ldots
$$
$f$ is in $\GTC$ because in this unique infinite branch 
the trace of the second argument crosses the right-hand-side of a $\cond$
infinitely many times.
Hence $f\in\CTlambda$.
From $f(x) \reduces \cond(x,f(\Succ x))$ we deduce that
 there is a infinite reduction sequence: \ \ 
 $
  f(x)\,\reduces\,\cond(x,f(\Succ(x))) \,
  \reduces\,
  \cond(x,\cond(\Succ(x),f(\Succ^2(x)))) \,
  \reduces\cdots
  $
  
  The limit normal form of $f(x)$
  is $\cond(x,\cond(\Succ(x),\cond(\Succ^2(x),\cdots)))$, 
  which is \emph{not} regular,
  since it has infinitely many different subterms $x,\Succ(x),\Succ^2(x),\ldots$. 
  Therefore the limit of $f(x)$ is not in $\CTlambda$, even though each term 
  finitely reachable from $f(x)$ is in $\CTlambda$, and $f$ is
  just the unary algorithm for the addition, because 
  
  $$
  f(\Succ^n(0))(\Succ^m(0)) \reduces \cond(\Succ^n(0),\cond(\Succ^{n+1}(0),\cond(\Succ^{n+2}(0),\cdots)))(\Succ^m(0)) \reduces (\Succ^{n+m})(0)
  $$
  
This is an unfortunate but common situation for the limit of a sequence
in a topology: finiteness of a notation (which in our case is expressed by 
\emph{regularity} for trees) is easily lost under limit.

%An example: suppose $T_{n}$ is the complete binary tree of height $n$ and
%$T_{\infty}$ is the complete infinite binary tree. Then
%$\lim_{n \rightarrow \infty} T_{n} = T_{\infty}$ and all $T_{n}$ are finite,
%yet $T_{\infty}$ is not finite.

%
%
%
%%%%%%%%%%%%%%%%%%%%%%%%%%%%%%%%%%%%%
% END MODIFICATION 11/02/2025 for the infinitely progressing trace
%%%%%%%%%%%%%%%%%%%%%%%%%%%%%%%%%%%%%
%
%
%


\subsection
{Closed Safe-Normal terms of $\CTlambda$ of type $\N$ are Numeral}
This property is not straightforward. 
There are closed normal terms of $\N$ in $\LAMBDA \setminus \CTlambda$, 
even regular terms, which are not numerals.
An example is $t = \cond(0, \lambda x^N.1)(t)$ 
(\cite{2021-Anupam-Das}, Remark 44, page 33). 
$t$ has a unique type $\N$. $t$ is normal,
because $t$ is not $0$ nor successor, 
therefore $t$ cannot be reduced. However $t$ is no numeral.

Remark that $t$ is regular: the subterms of $t$ are
$t$ itself and $\cond(0, \lambda x^N.1)$, $0$, $\lambda x^N.1$, $1$.
However, $t \not \in \GTC$, hence $t \not \in \CTlambda$. 
Indeed, the rightmost path of $t$ is an infinite path $t,t,t,t,\ldots$, 
never moving to the right-hand-side of a $\cond$,
and therefore never progressing.

We check that instead safe-normalization on closed terms of 
$\GTC$ of type $\N$ produces a numeral. For reason of efficiency,
we precise that safe-normalization can be done only contracting 
\emph{closed} redexes: {\bf no renaming is required}.
As an intermediate step in this proof we need the notion of \quotationMarks{sound} term.

\begin{definition}[Sound terms]
A term $t \in \CTlambda$ is sound if at least one of the following conditions holds.
\begin{enumerate}
\item
$t$ is open
\item
$t$ has some $0$-safe \emph{closed} redex
\item
$t=\lambda x.u$ or 
$t = \cond(f,g)$ for some $u$, $f$, $g$ and $t$ has an arrow type
\item
$t$ is a numeral
\end{enumerate}
\end{definition}

\begin{proposition}[Closed Safe-Normal terms of Type $\N$]
\label{proposition-closed-safe-normal-term-N}
\begin{enumerate}
\item
All $t \in \GTC$ are sound
\item
If $t \in \GTC$, $t$ is closed and $0$-safe-normal and $t:\N$ then $t \in \Num$ ($t$ is a numeral)
\end{enumerate}
\end{proposition}

%\begin{proof}
%\begin{enumerate}
%\item
%We assume that $t \in \GTC$ in order to prove that $t$ is sound.
%We argue by  induction on the size of the $0$-safe level of $t$: this size is finite because $t \in \GTC$. 
%We consider one case for each possible first symbol of $t$.
%
%\begin{enumerate}
%\item
%$t =x$. Then $t$ is open.
%\item
%$t=0$. Then $t$ is a numeral.
%\item
%$t = \Succ(u)$. Then $u:\N$ and $u$ has a smaller $0$-safe level, therefore $u$ is sound. If $u$ is open then
%$t$ is open, if $u$ has a $0$-safe redex then so does $t$, $u$ cannot have an arrow type, and if $u$
%is a numeral then $t$ is a numeral.
%\item
%$t=u(v)$. Then $u:A \rightarrow B$, $u:A$ and $u$, $v$ have a smaller $0$-safe level, therefore $u$,
%$v$ are sound. 
%
%If $u$ or $v$ are open then $t$ is open. 
%
%If $u$ or $v$ have a $0$-safe redex then so does $t$.
%
%Suppose $u = \lambda x.v, \cond(f,g)$. If $u = \lambda x.v$ then $t=u(v)$ is a $0$-safe $\beta$-redex.
%If $u = \cond(f,g)$ then $A = \N$ and $v:\N$. $v$ is not open nor has a a $0$-safe redex, and $v$ has no
%arrow type. The only possibility left is that $v$ is a numeral: we conclude that $t=\cond(f,g)(v)$ is a $\cond$-redex, therefore $t$ is sound.
%
%The only possibility left for $u$ is that it is a numeral. But this cannot be, because $u$ has an arrow type.
%
%\item
%$t = \lambda x.u$. Then $t$ is sound.
%
%\item
%$t = \cond(f,g)$. Then $t$ is sound.
%\end{enumerate}
%
%\item
%Assume that $t \in \GTC$, $t$ is closed and $0$-safe-normal and $t:\N$.
%Then $t$ is sound by the previous point and the only possibility left is that $t$ is 
%a numeral.
%\end{enumerate}
%\end{proof}