% !TEX root = main.tex

\section{Introduction} 
We consider $\LAMBDA$, the set of infinitary simply typed $\lambda$-terms 
with conditional, in the style for instance of 
Ariola, Bl\"{o}m, Klop, Kennaway, Sleep, de Vries 
\cite{ARIOLA1997154,10.1007/BFb0014548,KENNAWAY199793}).
The types of $\LAMBDA$ are: the atomic type $\N$, 
possibly type variables $\alpha, \beta, \ldots$, and all types 
$A \rightarrow B$ for any types $A$, $B$. 
Type variables are only used to provide examples of infinite
terms and play a minor role in our paper.

The terms of $\LAMBDA$  are all possibly well-typed infinite trees representing expressions 
defined with $0$, $\Succ $, $\ap$ (application), 
variables $x^T$ (with a type superscript $T$),  $\lambda$ (the binder for defining 
maps), and $\cond$, the arithmetic conditional (i.e., the test on zero). 
We will use infinite terms of $\LAMBDA$ as a simplified functional language
with fixed point, like in Clemens Grabmayer, Jan Rochel 
(\cite{Letrec,Letmu,JanRochelPhd2016}).
As usual, we add reduction rules transforming closed terms of type $\N$ 
into notations for natural numbers, therefore
the closed terms of $\LAMBDA$ represent partial functionals on $\N$.

%19:42 19/04/2024

In this paper will consider two sets of terms: 
\begin{enumerate}
\item
the set $\Reg$ of regular well-typed terms, those having a finite notation in
a circular syntax. This is the frame we work in.
%We claim that they are equivalent to the set of terms in G\"{o}del system $\systemT$ (\cite{GoedelSystemT}).  

\item
The subset $\CTlambda \subseteq \Reg$ of them, satisfying a 
condition called Global Trace Condition (GTC),
for which we prove our termination results.
The definition of $\CTlambda$ and proving its main properties is our contribution.
%$\GTC$ is a subset of $\LAMBDA$ providing a more narrow and perspicuous
%semantics to the finite graph notations in $\CTlambda$. 
\end{enumerate}

GTC has been introduced for infinitary proofs 
by J. Brotherston and A. Simpson
(\cite{BrotherstonPhd2006}, \cite{BrotherstonSimpson2011}), then adapted
to algebraic terms of G\"{o}del system $\systemT$ Das, Kuperberg, Pinault, Pous 
(\cite{2021-Anupam-Das,2021-Anupam-Das,DBLP:conf/fscd/000221,DBLP:conf/lics/Curzi022,DBLP:conf/csl/Curzi023,DBLP:conf/lics/Curzi023}).

%The descriptive power of the circular version of system $\systemT$ has been studied in detail.
%A. Das compared the descriptive power of circular terms of $\systemT$ 
%having degree $n$ type with the descriptive power of terms of original system $\systemT$,
%but having degree $n+1$ types. 
%He  proved in (\cite{2021-Anupam-Das}, \cite{DBLP:conf/fscd/000221})
%that the two set of terms define the same type $1$ functions. In the same papers he also proved 
%that the circular terms of $\systemT$ and the terms of original system $\systemT$ 
%define the same functionals at all types.
%
%G. Curzi and A. Das  also developed a circular version of Bellantoni-Cook system for 
%polynomial-time functions (\cite{DBLP:conf/lics/Curzi022}), 
%then some criterions for identifying polynomial-time functions
%in circular system $\systemT$ (\cite{DBLP:conf/csl/Curzi023}),
%and they extended circular syntax from $\systemT$ to all functions provably total in 
%$\Pi^1_2$-second order arithmetic (\cite{DBLP:conf/lics/Curzi023}).
%D. Kuperberg, L. Pinault and D. Pous (\cite{2021-Kuperberg-Pinault-Pous})
%proved that the contraction-free part of circular system $\systemT$
%captures exactly the primitive recursive functions.

%Global trace condition provides a sufficient and quite broad condition for the totality 
%of a functional. We introduce more subsets of $\LAMBDA$ which are only used 
%as intermediate steps in the definition of $\GTC$ and of the syntax 
%for $\CTlambda$.
%
%\begin{enumerate}
%\item
% $\WTyped \subseteq \LAMBDA$, the set of well-typed terms, is a
%set of terms having a unique simple type.  
%\item
%$\Reg$ is the set of all terms of $\LAMBDA$ which are possibly infinite
%regular trees (i.e., having finitely many subtrees), denoted by \emph{finite graphs}.
%\item 
%$\GTC \subseteq \WTyped$ will be defined as the set of well-typed circular 
%$\lambda$-terms satisfying the global trace condition (see later).  
%Terms of $\GTC$ will correspond to subset of the total functionals on $\N$. 
%\end{enumerate}

%We will prove that $\CTlambda$ is a decidable subset of the set $\Reg$ of infinite trees 
%having a finite graph notation.
$\CTlambda$ extends existing circular version of 
G\"{o}del system $\systemT$ with the binder $\lambda$. 
By introducing a circular syntax with binders, 
we hope to provide a circular syntax more familiar to researchers working in the
field of Type Theory. 
\\

We will prove for the extended circular syntax $\CTlambda$ the results already
known for the algebraic circular syntax for system $\systemT$:
that reductions not in the right-hand side of any $\cond$ (conditional)
strongly normalizes, and that they always reduce closed normal terms of type $\N$ to a numeral.
We will also prove that all infinite reductions have a limit, even if, as already remarked by 
Ariola and Klop (\cite{ARIOLA1997154}), not all limits are normal forms.



%14:57 17/04/2024
%21:08 19/04/2024

