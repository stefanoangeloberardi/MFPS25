% !TEX root = main.tex

\section{Introduction} 
We will introduce $\LAMBDA$, a set of infinite $\lambda$-terms we use to provide a \emph{semantics} for recursive definitions of simply typed $\lambda$-terms.
The types of $\LAMBDA$ are: the atomic type $\N$, 
possibly type variables $\alpha, \beta, \ldots$, and all types $A \rightarrow B$ for any types $A$, $B$. 
Type variables are only used to provide examples of infinite
terms and play a minor role in our paper.

The terms of $\LAMBDA$  are all possibly infinite trees representing expressions 
defined with $0$,$\Succ $,$\ap$ (application), 
variables $x^T$ (with a type superscript $T$),  $\lambda$ (the binder for defining 
maps), and $\cond$, the arithmetic conditional (i.e., the test on zero). 
We will use infinite terms of $\LAMBDA$ as a semantics for finite 
circular $\lambda$-terms.
If we have no free term variables and no type variables, 
in turn the trees in $\LAMBDA$ represent partial functionals on $\N$, 
provided we add reduction rules transforming closed terms of type $\N$ 
in notations for natural numbers.

%19:42 19/04/2024

In this paper will consider two sets of terms: 
\begin{enumerate}
\item
the set $\CTlambda$ of well-typed terms having a finite notation in
a circular syntax. We claim that they are equivalent to the set of terms in G\"{o}del system $\systemT$ (\cite{GoedelSystemT}).  

\item
The set of terms $\GTC$, satisfying a condition called global trace condition .
$\GTC$ is a subset of $\LAMBDA$ providing a more narrow and perspicuous
semantics to the finite graph notations in $\CTlambda$. 
\end{enumerate}

Global trace condition has been introduced by J. Brotherston and A. Simpson
(\cite{BrotherstonPhd2006}, \cite{BrotherstonSimpson2011}), then adapted
to algebraic terms of G\"{o}del system $\systemT$ by A. Das 
(\cite{2021-Anupam-Das}, published in \cite{DBLP:conf/fscd/000221}).
The descriptive power of the circular system $\systemT$ has been studied in detail.
A. Das compared the descriptive power of circular terms of $\systemT$ 
having degree $n$ type with the descriptive power of terms of original system $\systemT$,
but having degree $n+1$ types. 
He  proved in (\cite{2021-Anupam-Das}, \cite{DBLP:conf/fscd/000221})
that the two set of terms define the same type $1$ functions. In the same papers he also proved 
that the circular terms of $\systemT$ and the terms of original system $\systemT$ 
define the same functionals at all types.

G. Curzi and A. Das  also developed a circular version of Bellantoni-Cook system for 
polynomial-time functions (\cite{DBLP:conf/lics/Curzi022}), 
then some criterions for identifying polynomial-time functions
in circular system $\systemT$ (\cite{DBLP:conf/csl/Curzi023}),
and they extended circular syntax from $\systemT$ to all functions provably total in 
$\Pi^1_2$-second order arithmetic (\cite{DBLP:conf/lics/Curzi023}).
D. Kuperberg, L. Pinault and D. Pous (\cite{2021-Kuperberg-Pinault-Pous})
proved that the contraction-free part of circular system $\systemT$
captures exactly the primitive recursive functions.

%Global trace condition provides a sufficient and quite broad condition for the totality 
%of a functional. We introduce more subsets of $\LAMBDA$ which are only used 
%as intermediate steps in the definition of $\GTC$ and of the syntax 
%for $\CTlambda$.
%
%\begin{enumerate}
%\item
% $\WTyped \subseteq \LAMBDA$, the set of well-typed terms, is a
%set of terms having a unique simple type.  
%\item
%$\Reg$ is the set of all terms of $\LAMBDA$ which are possibly infinite
%regular trees (i.e., having finitely many subtrees), denoted by \emph{finite graphs}.
%\item 
%$\GTC \subseteq \WTyped$ will be defined as the set of well-typed circular 
%$\lambda$-terms satisfying the global trace condition (see later).  
%Terms of $\GTC$ will correspond to subset of the total functionals on $\N$. 
%\end{enumerate}

We will prove that $\CTlambda$ is a decidable subset of the set of infinite trees 
having a finite graph notation.
$\CTlambda$ is a new variant of the existing circular version of 
G\"{o}del system $\systemT$. Differently from all previous circular versions of 
$\systemT$, our system $\CTlambda$ uses binders instead of combinators. 
As we anticipated, circular syntax has the advange of writing terms using
finite simultaenous recursive definitions, allows for shorter denotation, while preserving decidability of termination. By introducing a circular syntax with binders, 
we hope to provide a circular syntax more familiar to researchers working in the
field of Type Theory.
\\

We will prove for the circular syntax $\CTlambda$
that reductions not in the right-hand side of any $\cond$ (conditional)
strong normalizes, and that they always reduce closed normal terms of type $\N$ to a numeral. In a future paper we plan to stress the relation with co-inductive definitions, to prove Church-Rosser, and to extend our results to a $\lambda$-
calculus including all Martin L\"{o}f finite Inductive Definitions, and not only
the set $\N$ of natural numbers.

%, and to prove \emph{strong} 
%normalization in the limit if we use reductions which are ``fair''.
%``Fair'' reductions can reduce \emph{inside} the right-hand side of some $\cond$, 
%but they never forget entirely the task of reducing \emph{outside} all such subterms.

In a future work, we also plan to prove that the closed terms $\CTlambda$ 
(those without free variables) represent exactly the total computable functionals 
definable in G\"{o}del system $\systemT$, using the techniques developed by 
Das for his version of cyclic terms.


%14:57 17/04/2024
%21:08 19/04/2024

