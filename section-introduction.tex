% !TEX root = main.tex

\section{Introduction} 
We will introduce $\LAMBDA$, a set of infinite $\lambda$-terms we use to provide a \emph{semantics} for recursive definitions of simply typed $\lambda$-terms.
The types of $\LAMBDA$ are: the atomic type $\N$, 
possibly type variables $\alpha, \beta, \ldots$, and all types $A \rightarrow B$ for any types $A$, $B$. 
Type variables are only used to provide examples of terms and play a minor role in our paper.

The terms of $\LAMBDA$  are all possibly infinite trees representing expressions defined with 
$0$,$\Succ $,$\ap$ (application), 
variables $x^T$ (with a type superscript $T$),  $\lambda$ (the binder for defining maps), 
and $\cond$, the arithmetic conditional (i.e., the test on zero). 
We will use infinite terms of $\LAMBDA$ to interpret finite 
circular $\lambda$-terms.
If we have no free term variables and no type variables, 
then the trees in $\LAMBDA$ represent partial functionals on $\N$, 
provided we add reduction rules transforming closed terms of type $\N$ in notations for natural numbers.

%19:42 19/04/2024

In this paper will consider two sets of terms: 
\begin{enumerate}
\item
the set $\CTlambda$ of well-typed terms having a finite notation in
a circular syntax. We claim that they are equivalent to the set of terms in G\"{o}del system $\systemT$.  
\item
The set of terms $\GTC$, satisfying a condition called global trace condition, which 
are infinite terms used to provide semantics to the finite notations in  $\CTlambda$ 
(and  $\systemT$, of course). Global trace condition provides a sufficient and quite 
broad condition for the totality of a functional.
\end{enumerate}
We introduce more sets of terms only used 
as intermediate steps in the definition of the semantic $\GTC$ and the syntax $\CTlambda$.

\begin{enumerate}
\item
 $\WTyped \subseteq \LAMBDA$, the set of well-typed terms, is the
of set terms having a unique simple type
\item
$\Reg$ is the set of terms of $\LAMBDA$ which are regular trees 
(i.e., having finitely many subtrees). They are possibly infinite terms which are finitely presented by a \emph{finite graph} of $\CTlambda$, possibly having cycles.
\item 
$\GTC \subseteq \WTyped$ will be defined as the set of well-typed circular 
$\lambda$-terms satisfying the global trace condition.  
Terms of $\GTC$ will correspond to subset of the total functionals. 
\end{enumerate}

We will prove that $\CTlambda$ is a decidable subset of $\Reg$.
$\CTlambda$ is a new variant of the existing circular version of 
G\"{o}del system $\systemT$. Differently from all previous circular versions of 
$\systemT$, our system $\CTlambda$ uses binders instead of combinators. 
As we anticipated, circular syntax has the advange of writing terms using
finite simultaenous recursive definitions,
Circular syntax allows for shorter denotation, while preserving decidability of 
termination. By introducing a circular syntax with binders, 
we hope to provide a circular syntax more familiar to researchers working in the
field of Type Theory.
\\

We will prove for the circular syntax $\CTlambda$
that reductions not in the right-hand side of any $\cond$ strong normalizes
and they always reduce closed normal terms of type $\N$ to a numeral.
In a future paper we plan to stress the relation with co-inductive definitions
to prove Church-Rosser.
%, and to prove \emph{strong} 
%normalization in the limit if we use reductions which are ``fair''.
%``Fair'' reductions can reduce \emph{inside} the right-hand side of some $\cond$, 
%but they never forget entirely the task of reducing \emph{outside} all such subterms.

In a future work, we also plan to prove that the closed terms $\CTlambda$ 
(those without free variables) represent exactly the total computable functionals 
definable in G\"{o}del system $\systemT$, using the techniques developed by 
Anupam for his version of cyclic terms.


%14:57 17/04/2024
%21:08 19/04/2024

