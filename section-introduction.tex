% !TEX root = main.tex

\section{Introduction}

We consider $\LAMBDA$, the set of infinitary simply typed $\lambda$-terms 
with conditional on zero, in the style for instance of 
Ariola, Bl\"{o}m, Klop, Kennaway, Sleep, de Vries 
(\cite{ARIOLA1997154,10.1007/BFb0014548,KENNAWAY199793}).

The types of $\LAMBDA$ are all simple types obtained from the atomic type $\N$
and type variables.
The terms of $\LAMBDA$  are all possibly infinite trees representing expressions 
defined with $0$, $\Succ $, $\ap$ (application), 
variables $x^T$ (with a type superscript $T$),  $\lambda$ (the binder for defining 
maps), and $\cond$, the arithmetic conditional (i.e., the test on zero, with predecessor included). 
We imagine that the infinite terms of $\LAMBDA$ describe
a simplified functional language with fixed point, like the functional language considered by
 Clemens Grabmayer, Jan Rochel (\cite{Letrec,Letmu,JanRochelPhd2016}).
As usual, we add reduction rules with the goal of transforming closed terms of type $\N$ 
into notations for natural numbers. Thus, 
the closed terms of $\LAMBDA$ can be interpreted as partial functionals on $\N$.
In the general case a reduction on a closed terms of type $\N$ can continue forever,
this is why without further conditions we only obtain partial functionals.

%19:42 19/04/2024

In this paper we will mainly consider regular terms, namely, possibly infinite $\lambda$-terms 
having a finite number of distinct subterms. This is the frame we work in.
We introduce a set of terms we call $\CTlambda$, consisting of all 
regular terms satisfying a condition called Global Trace Condition (GTC).
The definition of $\CTlambda$ and a termination result for it is our contribution.

GTC has been introduced for infinitary proofs 
by Brotherston and Simpson
(\cite{BrotherstonPhd2006}, \cite{BrotherstonSimpson2011}), then adapted
to algebraic terms of G\"{o}del system $\systemT$ by Das, Kuperberg, Pinault, Pous 
(\cite{2021-Anupam-Das,DBLP:conf/fscd/000221,DBLP:conf/lics/Curzi022,DBLP:conf/csl/Curzi023,DBLP:conf/lics/Curzi023}).

Our contribution is to extend these circular versions of G\"{o}del system $\systemT$ 
(\cite{GoedelSystemT}) by adding an explicit use of the binder $\lambda$. Adding $\lambda$,  
we hope to provide a circular syntax more familiar to researchers working in the
field of Type Theory. % We call $\CTlambda$ our extended circular syntax.

We will prove for $\CTlambda$ (with binders) a result which was only
known for the \emph{algebraic} circular syntax for system $\systemT$:
that reductions not in the second argument of any $\cond$ (conditional)
strongly normalizes, and that they always reduce closed normal terms of type 
$\N$ to a numeral.

We also quote a second contribution. We extend to the case of binder $\lambda$ the
characterization of the limit reduction obtained 
by Das, Kuperberg, Pinault, Pous 
(\cite{2021-Anupam-Das,DBLP:conf/fscd/000221,DBLP:conf/lics/Curzi022,DBLP:conf/csl/Curzi023,DBLP:conf/lics/Curzi023})
for the algebraic circular version of 
G\"{o}del system $\systemT$. Namely, we prove that all infinite reductions on $\CTlambda$
have a limit, which informally can be described as follows. 
In each infinite reduction sequence
the first symbol of the $\lambda$-term eventually stabilizes, then the
first symbol of each immediate subterm eventually stabilizes, and so forth, recursively.

The notion of limit reduction in $\CTlambda$
could be considered similar to the notion of limit reduction for finite $\lambda$-terms obtained
through B\"{o}hm trees (\cite{Barendregt1984}, Def. 10.1.4). 
However, we stress that there is an important difference: 
in $\CTlambda$ no \quotationMarks{undefined} subterm is required
in order to represent the limit of a reduction, while B\"{o}hm trees 
can have undefined subtrees. 

This is the plan of the paper. In Section~\ref{section-infinite-lambda}, we introduce
the set $\LAMBDA$ of our infinite $\lambda$-calculus,
including its types, terms, typing rules, and reduction rules.
Section~\ref{section-trace-infinite-lambda-terms} and Section~\ref{section-circular-system-CTlambda}
introduce the notion of traces and the global trace condition for $\LAMBDA$, respectively.
Section~\ref{section-subject-reduction} mentions the subject reduction
for terms of $\LAMBDA$ with GTC. 
Section~\ref{section-finite-safe-reductions} introduces the notion of $n$-safe-reductions
that compute finite approximations of the limit normal forms of infinite reductions, 
and also shows that any closed $\CTlambda$ term of type $\N$ is reduced to a numeral
by reducing in any way with $0$-safe-reductions. 
Section~\ref{section-comparison} compares our work with some related work. 
Section~\ref{section-conclusion} concludes the current paper, mentioning our plan 
for further works. 


%14:57 17/04/2024
%21:08 19/04/2024

