% !TEX root = main.tex

\section{Introduction} 
We will introduce $\LAMBDA$, a set of infinite $\lambda$-terms with a circular syntax.
The types of $\LAMBDA$ are: the atomic type $\N$, 
possibly type variables $\alpha, \beta, \ldots$, and all types $A \rightarrow B$ for any types $A$, $B$. 
Type variables are only used to provide examples of terms and play a minor role in our paper.

The terms of $\LAMBDA$  are all possibly infinite trees representing expressions defined with 
$0$,$\Succ $,$\ap$ (application), 
variables $x^T$ (with a type superscript $T$),  $\lambda$ (the binder for defining maps), 
and $\cond$, the arithmetic conditional (i.e., the test on zero). 
If we have no term variables and no type variables, 
then the trees in $\LAMBDA$ represent partial functionals on $\N$, 
provided we add reduction rules transforming closed terms of type $\N$ in notations for natural numbers.

%19:42 19/04/2024

In this paper will consider two sets of terms: 
\begin{enumerate}
\item
the set $\CTlambda$ of well-typed terms in
a circular syntax, which are equivalent to the set of terms in G\"{o}del system $\systemT$.  
\item
The set of terms $\GTC$, satisfying a condition called global trace condition, which are possibly non-recursive
terms used to provide semantics for  $\CTlambda$ (and  $\systemT$, of course)
\end{enumerate}
We introduce more sets of terms only used 
as intermediate steps in the definition of the semantic $\GTC$ and the syntax $\CTlambda$.

\begin{enumerate}
\item
 $\WTyped \subseteq \LAMBDA$, the set of well-typed terms, is the
of set terms having a unique type
\item
$\Reg$ is the set of terms of $\LAMBDA$ which are regular trees (i.e., having finitely
many subtrees). They are possibly infinite terms which are finitely presented 
by a finite graph possibly having cycles.
\item 
$\GTC \subseteq \WTyped$ will be defined as the set of well-typed circular 
$\lambda$-terms satisfying the global trace condition (and possibly non-recursive, as we said). %and regular. 
Terms of $\GTC$ denote total functionals. 
\end{enumerate}

We will prove that $\CTlambda$ is a decidable subset of $\Reg$.
$\CTlambda$ is a new variant of the existing circular version of G\"{o}del system $\systemT$. 
Differently from all previous circular versions of $\systemT$, our system $\CTlambda$
uses binders instead of combinators. 
As we anticipated, 
 circular syntax has the advange of writing much shorter terms while preserving decidability of termination.
Besides, by introducing a circular syntax with binders, we hope to provide 
a circular syntax more familiar to researchers working in the field of Type Theory.
\\

We will prove the expected results for the circular syntax $\CTlambda$:
strong normalization for reductions not in the right-hand side of any $\cond$ and Church-Rosser. 
We will prove normalization in the limit if we use reductions which are ``fair'':
fair reductions can reduce \emph{inside} the right-hand side of some $\cond$, but they never forget entirely 
the task of reducing \emph{outside} all such subterms.

Eventually, we will prove that the closed terms $\CTlambda$ (those without free variables)
represent exactly the total computable functionals definable in G\"{o}del system $\systemT$.


%14:57 17/04/2024
%21:08 19/04/2024

